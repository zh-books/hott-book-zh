\chapter{范畴论 (Category theory)}
\label{cha:category-theory}

在数学的各个分支中,范畴论可能是最不适合基于集合论基础的学科之一。
一个问题是,大多数范畴论的内容在弱于等同性的“相同”概念下是不变的,例如范畴中的同构或者范畴之间的等价,而集合论无法捕捉这一点。
但是,对于类型论中的等同性,无效性公理 (univalence axiom) 通过将等同性与等价性等同解决了类似的问题。
因此,在无效性基础 (univalent foundations) 中,考虑一种“范畴”概念是合理的,在这种范畴中,对象的等同性与同构被以类似的方式识别。

忽略大小问题,在基于集合的数学中,范畴由一个对象的\emph{集合} $A_0$ 和对于每个 $x,y\in A_0$ 的\emph{集合} $\hom_A(x,y)$ 组成。
在无效性基础下,一个“朴素的”范畴定义只是模仿这个结构,用一个\emph{类型}的对象和\emph{类型}的态射。
如果我们允许这些类型包含任意的高阶同伦,则应当引入更高的连贯条件,从而导向某种 $(\infty,1)$-范畴的概念,\index{.infinity1-category@$(\infty,1)$-范畴}%
但目前我们的目标更加温和。我们只考虑 1-范畴,因此我们将 $\hom_A(x,y)$ 的类型限制为集合,即 0-类型。如果我们不施加进一步的条件,我们将这种概念称为\emph{预范畴}。

如果我们添加条件,使得对象类型 $A_0$ 是一个集合,那么我们得到一个行为与传统集合论定义相似的定义。
我们称这种概念为\emph{严格范畴} (strict category)。
\index{严格!范畴 (strict category)}%
或者,我们可以要求无效性公理的广义版本,将 $(x=_{A_0} y)$ 与 $x$ 到 $y$ 的同构类型 $\mathsf{iso}(x,y)$ 识别。
由于我们认为后一种选择通常是“正确的”定义,因此我们简单地称它为\emph{范畴}。

一个关于这三种范畴概念的区别的好例子是“每个满忠实且本质满的函子是范畴等价”的命题,在经典基于集合的范畴论中,该命题等价于选择公理。
\index{数学!经典 (mathematics!classical)}%
\index{公理!选择 (axiom!of choice)}%
\index{经典!范畴论 (classical!category theory)}%
\begin{enumerate}
  \item 对于严格范畴,这一命题仍然等价于选择公理。
  \item 对于预范畴,没有一致的选择公理可以使其成立。
  \item 对于范畴,它可以\emph{不使用}任何选择公理的情况下被证明。
\end{enumerate}
我们将在本章中证明最后一个命题,并讨论范畴的其他令人愉快的性质,例如等价范畴是相等的(作为范畴类型的元素)。
我们还将描述一种将预范畴 $A$“饱和”成范畴 $\widehat A$ 的通用方式,我们称之为它的\emph{Rezk 完成},\index{完成!Rezk (completion!Rezk)}%
尽管它也可以合理地称为\emph{栈完成}(见注释)。

Rezk 完成还进一步阐明了范畴等价的概念。
例如,函子 $A \to \widehat{A}$ 始终是满忠实且本质满的,因此是“弱等价”。
因此,一个预范畴是范畴的必要和充分条件是它“视”所有满忠实且本质满的函子为等价;因此我们的“范畴”概念已经内在于“满忠实且本质满函子”的概念之中。

我们假定读者对经典范畴论有一定的基本了解。 \index{经典!范畴论 (classical!category theory)}
请记住,每当我们写到 \type 时,它表示某个类型的宇宙,但在不同情况下可能是不同的宇宙;我们所说的一切都对任何一致的宇宙层次的选择成立。\index{宇宙层次 (universe level)}
我们将使用 \cref{cha:typetheory,cha:basics} 中的同伦类型论的基本概念以及 \cref{cha:logic} 中的命题截断,但不会涉及 \cref{part:foundations} 的大部分内容,除非我们将在第二种 Rezk 完成的构造中使用高阶归纳类型。

\section{范畴与预范畴 (Categories and precategories)}
\label{sec:cats}

在经典数学中,有许多等价的范畴定义。
在我们的情况下,由于我们有依赖类型,因此将箭头作为对象的类型族是很自然的。
这与范畴论中 Hom-类型的使用方式相匹配:我们在比较两个箭头之前从未考虑过它们的域和陪域是否一致。
此外,显然,对于 1-范畴理论,所有的 Hom-类型都应该是集合。
这引导我们到以下定义。

\begin{defn}\label{ct:precategory}
一个\define{预范畴 (precategory)}\indexdef{预范畴 (precategory)}$A$ 包含以下内容。
\begin{enumerate}
  \item 一个类型 $A_0$,其元素称为\define{对象 (objects)}。%
  \indexdef{对象!在(预)范畴中 (object!in a (pre)category)}
  我们记 $a:A$ 表示 $a:A_0$。
  \item 对于每个 $a,b:A$,有一个集合 $\hom_A(a,b)$,其元素称为\define{箭头 (arrows)} 或\define{态射 (morphisms)}。%
  \indexsee{箭头}{态射}%
  \indexdef{态射!在(预)范畴中 (morphism!in a (pre)category)}%
  \indexdef{同态集 (hom-set)}%
  \item 对于每个 $a:A$,有一个态射 $1_a:\hom_A(a,a)$,称为\define{恒等态射 (identity morphism)}。%
  \indexdef{恒等!在(预)范畴中 (identity!morphism in a (pre)category)}
  \item 对于每个 $a,b,c:A$,有一个函数%
  \indexdef{态射的复合在(预)范畴中 (composition!of morphisms in a (pre)category)}
  \[  \hom_A(b,c) \to \hom_A(a,b) \to \hom_A(a,c) \]
  称为\define{复合 (composition)},记为中缀形式 $g\mapsto f\mapsto g\circ f$,有时简单记为 $gf$。
  \item 对于每个 $a,b:A$ 和 $f:\hom_A(a,b)$,我们有 $\id f {1_b\circ f}$ 和 $\id f {f\circ 1_a}$。
  \item 对于每个 $a,b,c,d:A$ 和
  \begin{equation*}
    f:\hom_A(a,b), \qquad
    g:\hom_A(b,c), \qquad
    h:\hom_A(c,d),
  \end{equation*}
  我们有 $\id {h\circ (g\circ f)}{(h\circ g)\circ f}$。
\end{enumerate}
\end{defn}

预范畴的一个问题是,对于对象 $a,b:A$,我们有两种可能不同的“相同”概念。
一方面,我们有类型 $(\id[A_0]{a}{b})$。
但另一方面,有标准的范畴论中的\emph{同构 (isomorphism)}的概念。

\begin{defn}\label{ct:isomorphism}
态射 $f:\hom_A(a,b)$ 是一个\define{同构 (isomorphism)}\indexdef{同构!在(预)范畴中 (isomorphism!in a (pre)category)}%
如果存在态射 $g:\hom_A(b,a)$ 使得 $\id{g\circ f}{1_a}$ 和 $\id{f\circ g}{1_b}$。
我们记 $a\cong b$ 表示这种同构的类型。
\end{defn}

\begin{lem}\label{ct:isoprop}
对于任意 $f:\hom_A(a,b)$,类型“$f$ 是同构”是一个单命题。
因此,对于任意 $a,b:A$,类型 $a\cong b$ 是一个集合。
\end{lem}
\begin{proof}
  假设给定 $g:\hom_A(b,a)$ 和 $\eta:(\id{1_a}{g\circ f})$ 以及 $\epsilon:(\id{f\circ g}{1_b})$,类似地有 $g'$,$\eta'$ 和 $\epsilon'$。
  我们必须证明 $\id{(g,\eta,\epsilon)}{(g',\eta',\epsilon')}$。
  但由于所有同态集都是集合,因此它们的等同性类型是单命题,所以只需证明 $\id g {g'}$。
  为此我们有
  \[g' = 1_a\circ g' = (g\circ f)\circ g' = g\circ (f\circ g') = g\circ 1_b = g\]
  使用 $\eta$ 和 $\epsilon'$。
\end{proof}

\symlabel{ct:inv}
\index{逆在(预)范畴中 (inverse!in a (pre)category)}%
如果 $f:a\cong b$,我们记 $\inv f$ 表示它的逆,由 \cref{ct:isoprop} 唯一确定。

在预范畴中,这两种相同性概念之间的唯一关系如下。

\begin{lem}[\textsf{idtoiso}]\label{ct:idtoiso}
如果 $A$ 是一个预范畴,$a,b:A$,则
\[(\id a b)\to (a \cong b)\]。
\end{lem}
\begin{proof}
  通过等同性归纳,我们可以假设 $a$ 和 $b$ 是相同的。
  但是在这种情况下,我们有 $1_a:\hom_A(a,a)$,它显然是一个同构。
\end{proof}

显然,这种情况类似于促使我们引入无效性公理的问题。
实际上,我们有以下例子:

\begin{eg}\label{ct:precatset}
\index{集合 (set)}%
存在一个预范畴 \uset,其对象的类型是 \set,并且 $\hom_{\uset}(A,B) \defeq (A\to B)$。
恒等态射是恒等函数,复合是函数复合。
对于这个预范畴,\cref{ct:idtoiso} 等同于(限制到集合的)来自 \cref{sec:compute-universe} 中的 $\idtoeqv$。

当然,更准确地说,我们应该称这个范畴为 $\uset_\UU$,因为它的对象仅是相对于某个宇宙 \UU 的\emph{小集合}。\index{小!集合 (small!set)}%
\end{eg}

因此,引入以下定义是合乎逻辑的。

\begin{defn}\label{ct:category}
一个\define{范畴 (category)}\indexdef{范畴 (category)}是一个预范畴,使得对于所有 $a,b:A$,\cref{ct:idtoiso} 中的函数 $\idtoiso_{a,b}$ 是一个等价。
\end{defn}

特别是,在一个范畴中,如果 $a\cong b$,那么 $a=b$。

\begin{eg}\label{ct:eg:set}
\index{无效性公理 (univalence axiom)}%
无效性公理立即蕴含 \uset 是一个范畴。
还可以通过无效性公理证明,任何集合层次的结构(如群、环、拓扑空间等)的预范畴都是一个范畴;见 \cref{sec:sip}。
\end{eg}

我们还注意到以下内容。

\begin{lem}\label{ct:obj-1type}
在一个范畴中,对象的类型是一个 1-类型。
\end{lem}
\begin{proof}
  证明对于任意 $a,b:A$,类型 $\id a b$ 是一个集合即可。
  但 $\id a b$ 等价于 $a \cong b$,而后者是一个集合。
\end{proof}

\symlabel{isotoid}
我们记 $\isotoid$ 为 \cref{ct:idtoiso} 中从 $(a\cong b)$ 到 $(\id a b)$ 的逆。
以下关系非常重要。

\begin{lem}\label{ct:idtoiso-trans}
对于 $p:\id a a'$ 和 $q:\id b b'$ 以及 $f:\hom_A(a,b)$,我们有
\begin{equation}\label{ct:idtoisocompute}
\id{\trans{(p,q)}{f}}
{\idtoiso(q)\circ f \circ \inv{\idtoiso(p)}}。
\end{equation}
\end{lem}
\begin{proof}
  通过归纳法,我们可以假设 $p$ 和 $q$ 分别是 $\refl a$ 和 $\refl b$。
  在这种情况下,\eqref{ct:idtoisocompute} 的左边简化为 $f$。
  但根据定义,$\idtoiso(\refl a)$ 是 $1_a$,$\idtoiso(\refl b)$ 是 $1_b$,因此 \eqref{ct:idtoisocompute} 的右边为 $1_b\circ f\circ 1_a$,这等于 $f$。
\end{proof}

类似地,我们可以证明
\begin{gather}
  \id{\idtoiso(\rev p)}{\inv {(\idtoiso(p))}}\\
  \id{\idtoiso(p\ct q)}{\idtoiso(q)\circ \idtoiso(p)}\\
  \id{\isotoid(f\circ e)}{\isotoid(e)\ct \isotoid(f)}
\end{gather}
等等。

\begin{eg}\label{ct:orders}
一个预范畴,其中每个集合 $\hom_A(a,b)$ 是单命题等价于一个类型 $A_0$,配有一个反身 ($a\le a$) 和传递 ($a\le b$ 且 $b\le c$,则 $a\le c$) 的单命题关系“$\le$”。
我们称这种结构为\define{预序 (preorder)}。\indexdef{预序 (preorder)}

在预序中,证据 $f: a\le b$ 是一个同构当且仅当存在某种证据 $g: b\le a$。
因此,$a\cong b$ 是单命题,即 $a\le b$ 且 $b\le a$。
因此,一个预序 $A$ 是范畴当且仅当 (1) 每个类型 $a=b$ 是单命题,(2) 对于任何 $a,b:A_0$ 存在一个函数 $(a\cong b) \to (a=b)$。
换句话说,$A_0$ 必须是一个集合,且 $\le$ 必须是反对称的\index{关系!反对称 (relation!antisymmetric)}(如果 $a\le b$ 且 $b\le a$,则 $a=b$)。
我们称这种结构为\define{(偏序) (partial) order} 或\define{偏序集 (poset)}。\indexdef{偏序 (partial order)}%
\indexdef{偏序集 (poset)}%
\end{eg}

\begin{eg}\label{ct:gaunt}
如果 $A$ 是一个范畴,那么 $A_0$ 是一个集合当且仅当对于任意 $a,b:A_0$,类型 $a\cong b$ 是单命题。
给定 $A$ 是一个范畴,这等价于说每个自同构在 $A$ 中都是恒等箭头。另一方面,如果 $A$ 是一个预范畴且 $A_0$ 是一个集合,那么 $A$ 是范畴当且仅当它是骨架 (skeletal)\index{数学!经典 (mathematics!classical)}(任何两个同构的对象是相等的)\emph{并且}每个自同构是恒等箭头。
这种类型的范畴有时被称为\define{gaunt} 范畴~\cite{bsp12infncats}。\indexdef{范畴!gaunt (category!gaunt)}%
\indexdef{gaunt 范畴 (gaunt category)}%
\index{骨架范畴 (skeletal category)}%
\index{范畴!骨架 (category!skeletal)}%
对于我们的范畴,没有真正的“骨架性”概念,除非认为 \cref{ct:category} 本身是这样的。
\end{eg}

\begin{eg}\label{ct:discrete}
对于任意 1-类型 $X$,存在一个范畴,其对象类型为 $X$,且 $\hom(x,y) \defeq (x=y)$。
如果 $X$ 是一个集合,我们称其为\define{离散范畴 (discrete category)}。\indexdef{范畴!离散 (category!discrete)}%
\indexdef{离散!范畴 (discrete!category)}%
通常我们称其为\define{群组 (groupoid)}(见 \cref{ct:groupoids})。
\end{eg}

\begin{eg}\label{ct:fundgpd}
对于\emph{任意}类型 $X$,存在一个预范畴,其对象类型为 $X$,且 $\hom(x,y) \defeq \pizero{x=y}$。
复合运算
\[ \pizero{y=z} \to \pizero{x=y} \to \pizero{x=z} \]
通过同伦连接从复合 $(y=z)\to(x=y)\to(x=z)$ 归纳定义。
我们称其为 $X$ 的\define{基本预群组 (fundamental pregroupoid)}。\indexdef{基本!预群组 (fundamental!pregroupoid)}%
\indexsee{预群组,基本 (pregroupoid, fundamental)}{基本预群组 (fundamental pregroupoid)}%
(实际上,我们在 \cref{sec:van-kampen} 中已经遇到了它;另见 \cref{ex:rezk-vankampen}。)
\end{eg}

\begin{eg}\label{ct:hoprecat}
存在一个预范畴,其对象类型是 \type,且 $\hom(X,Y) \defeq \pizero{X\to Y}$,复合通过同伦连接从普通复合 $(Y\to Z) \to (X\to Y) \to (X\to Z)$ 归纳定义。
我们称其为\define{类型的同伦预范畴 (homotopy precategory of types)}。\indexdef{类型的预范畴 (precategory!of types)}%
\index{同伦!类型范畴 (homotopy!category of types)}%
\end{eg}

\begin{eg}\label{ct:rel}
令 \urel 为以下预范畴:
\begin{itemize}
  \item 其对象是集合。
  \item $\hom_{\urel}(X,Y) = X\to Y\to \prop$。
  \item 对于集合 $X$,我们有 $1_X(x,x') \defeq (x=x')$。
  \item 对于 $R:\hom_{\urel}(X,Y)$ 和 $S:\hom_{\urel}(Y,Z)$,它们的复合定义为
  \[ (S\circ R)(x,z) \defeq \Brck{\sm{y:Y} R(x,y) \times S(y,z)}。\]
\end{itemize}
假设 $R:\hom_{\urel}(X,Y)$ 是一个同构,其逆为 $S$。
我们观察到以下几点。
\begin{enumerate}
  \item 如果 $R(x,y)$ 且 $S(y',x)$,则 $(R\circ S)(y',y)$,从而 $y'=y$。
  类似地,如果 $R(x,y)$ 且 $S(y,x')$,则 $x=x'$。\label{item:rel1}
  \item 对于任意 $x$,我们有 $x=x$,因此 $(S\circ R)(x,x)$。
  因此,仅存在某个 $y:Y$ 使得 $R(x,y)$ 且 $S(y,x)$。\label{item:rel2}
  \item 假设 $R(x,y)$。
  根据~\ref{item:rel2},仅存在某个 $y'$ 使得 $R(x,y')$ 且 $S(y',x)$。
  但随后根据~\ref{item:rel1},仅 $y'=y$,因此由于 $Y$ 是集合,所以 $y'=y$。
  因此,通过沿此等同性传递 $S(y',x)$,我们有 $S(y,x)$。
  结论是 $R(x,y)\to S(y,x)$。
  类似地,$S(y,x) \to R(x,y)$。\label{item:rel3}
  \item 如果 $R(x,y)$ 且 $R(x,y')$,则根据~\ref{item:rel3},$S(y',x)$,因此根据~\ref{item:rel1},$y=y'$。
  因此,对于任意 $x$,至多存在一个 $y$ 使得 $R(x,y)$。
  根据~\ref{item:rel2},仅存在这样的 $y$,因此存在这样的 $y$。
\end{enumerate}
结论是,如果 $R:\hom_{\urel}(X,Y)$ 是同构,则对于每个 $x:X$,存在且仅存在一个 $y:Y$ 使得 $R(x,y)$,反之亦然。
因此,存在一个函数 $f:X\to Y$,将每个 $x$ 映射到此 $y$,该函数是等价的;因此 $X=Y$。
经过更多的推导,我们可以得出 \urel 是一个范畴。
\end{eg}

我们现在可能将自己限制为仅考虑范畴而非预范畴。
相反,我们将对预范畴和范畴进行许多概念的开发,以强调范畴相比于预范畴以及经典\index{数学!经典 (mathematics!classical)}数学中的普通范畴如何更加良好。

我们还将在 \crefrange{sec:strict-categories}{sec:dagger-categories} 中看到,在稍微异质的背景下,某些类型的预范畴除了范畴之外还有其他用途,每种预范畴以不同的方式“修复”对象的等同性。
这进一步强调了预范畴的“预”性:它们是从中可以定义多种重要范畴结构的原材料。
\section{函子与变换 (Functors and transformations)}
\label{sec:transfors}

以下定义非常明显,无需修改。

\begin{defn}\label{ct:functor}
设 $A$ 和 $B$ 是前范畴 (precategories)。
一个\define{函子 (functor)} $F:A\to B$ 由以下部分组成:
\begin{enumerate}
  \item 一个函数 $F_0:A_0\to B_0$,通常也记作 $F$。
  \item 对每个 $a,b:A$,一个函数 $F_{a,b}:\hom_A(a,b) \to \hom_B(Fa,Fb)$,通常也记作 $F$。
  \item 对每个 $a:A$,我们有 $\id{F(1_a)}{1_{Fa}}$。
  \item 对每个 $a,b,c:A$ 以及 $f:\hom_A(a,b)$ 和 $g:\hom_A(b,c)$,我们有
  \[\id{F(g\circ f)}{Fg\circ Ff}。\]\label{ct:functor:comp}
\end{enumerate}
\end{defn}

注意,通过同一性归纳 (induction on identity),函子还保持 $\idtoiso$。

\begin{defn}\label{ct:nattrans}
对于函子 $F,G:A\to B$,一个\define{自然变换 (natural transformation)} $\gamma:F\to G$ 由以下部分组成:
\begin{enumerate}
  \item 对每个 $a:A$,一个态射 $\gamma_a:\hom_B(Fa,Ga)$ (称为“分量”)。
  \item 对每个 $a,b:A$ 和 $f:\hom_A(a,b)$,我们有 $\id{Gf\circ \gamma_a}{\gamma_b\circ Ff}$ (称为“自然性公理”)。
\end{enumerate}
\end{defn}

由于每个类型 $\hom_B(Fa,Gb)$ 是一个集合,因此它的同一性类型 (identity type) 是一个纯命题 (mere proposition)。
因此,自然性公理是一个纯命题,所以自然变换的同一性由它们的分量的同一性决定。
特别地,对于任意 $F$ 和 $G$,从 $F$ 到 $G$ 的自然变换的类型也是一个集合。

同样,函子的同一性由函数 $A_0\to B_0$ 的同一性(以及在此基础上运输的同源集合上的相应函数)决定。

\begin{defn}\label{ct:functor-precat}
对于前范畴 $A,B$,存在一个称为\define{函子前范畴 (functor precategory)} 的前范畴 $B^A$,其定义如下:
\begin{itemize}
  \item $(B^A)_0$ 是从 $A$ 到 $B$ 的函子的类型。
  \item $\hom_{B^A}(F,G)$ 是从 $F$ 到 $G$ 的自然变换的类型。
\end{itemize}
\end{defn}
\begin{proof}
  我们定义 $(1_F)_a\defeq 1_{Fa}$。
  自然性遵循前范畴的单位公理。
  对于 $\gamma:F\to G$ 和 $\delta:G\to H$,我们定义 $(\delta\circ\gamma)_a\defeq \delta_a\circ \gamma_a$。
  自然性遵循结合性。
  同样,$B^A$ 的单位和结合性公理也遵循 $B$ 的单位和结合性公理。
\end{proof}

\begin{lem}\label{ct:natiso}
一个自然变换 $\gamma:F\to G$ 如果且仅如果每个 $\gamma_a$ 在 $B$ 中是一个同构,那么它在 $B^A$ 中是一个同构。
\end{lem}
\begin{proof}
  如果 $\gamma$ 是一个同构,那么我们有 $\delta:G\to F$ 作为它的逆变换。
  根据在 $B^A$ 中的合成定义,$(\delta\gamma)_a\jdeq \delta_a\gamma_a$,同样地 $(\gamma\delta)_a\jdeq \gamma_a\delta_a$。
  因此,$\id{\delta\gamma}{1_F}$ 和 $\id{\gamma\delta}{1_G}$ 意味着 $\id{\delta_a\gamma_a}{1_{Fa}}$ 和 $\id{\gamma_a\delta_a}{1_{Ga}}$,因此 $\gamma_a$ 是一个同构。

  反过来,假设每个 $\gamma_a$ 是一个同构,设它的逆为 $\delta_a$。
  我们定义一个自然变换 $\delta:G\to F$,其分量为 $\delta_a$;对于自然性公理,我们有
  \[ Ff\circ \delta_a = \delta_b\circ \gamma_b\circ Ff \circ \delta_a = \delta_b\circ Gf\circ \gamma_a\circ \delta_a = \delta_b\circ Gf. \]
  现在,由于自然变换的合成和同一性是由它们的分量决定的,我们有 $\id{\gamma\delta}{1_G}$ 和 $\id{\delta\gamma}{1_F}$。
\end{proof}

以下结果是基础性的。

\begin{thm}\label{ct:functor-cat}
如果 $A$ 是一个前范畴而 $B$ 是一个范畴 (category),那么 $B^A$ 是一个范畴。
\end{thm}
\begin{proof}
  令 $F,G:A\to B$;我们必须证明 $\idtoiso:(\id{F}{G}) \to (F\cong G)$ 是一个等价。

  为了给它一个逆,我们假设 $\gamma:F\cong G$ 是一个自然同构。
  然后对于任意 $a:A$,我们有一个同构 $\gamma_a:Fa \cong Ga$,因此一个同一性 $\isotoid(\gamma_a):\id{Fa}{Ga}$。
  通过函数扩展性 (function extensionality),我们有一个同一性 $\bar{\gamma}:\id[(A_0\to B_0)]{F_0}{G_0}$。

  现在,由于函子的最后两个公理是纯命题,为了证明 $\id{F}{G}$,我们只需证明对于任意 $a,b:A$,函数
  \begin{align*}
    F_{a,b}&:\hom_A(a,b) \to \hom_B(Fa,Fb)\mathrlap{\qquad\text{和}}\\
    G_{a,b}&:\hom_A(a,b) \to \hom_B(Ga,Gb)
  \end{align*}
  在沿着 $\bar\gamma$ 运输时变得相等。
  通过函数扩展性的计算,当应用于 $a$ 时,$\bar\gamma$ 变得等于 $\isotoid(\gamma_a)$。
  但通过 \cref{ct:idtoiso-trans},沿 $\isotoid(\gamma_a)$ 和 $\isotoid(\gamma_b)$ 运输 $Ff:\hom_B(Fa,Fb)$ 等于复合 $\gamma_b\circ Ff\circ \inv{(\gamma_a)}$,这通过 $\gamma$ 的自然性等于 $Gf$。

  这完成了从 $(F\cong G) \to (\id F G)$ 的函数的定义。
  现在考虑复合
  \[ (\id F G) \to (F\cong G) \to (\id F G). \]
  由于同源集 (hom-sets) 是集合,它们的同一性类型是纯命题,因此为了证明两个同一性 $p,q:\id F G$ 是相等的,只需证明 $\id[\id{F_0}{G_0}]{p}{q}$。
  但在 $\bar\gamma$ 的定义中,如果 $\gamma$ 是 $\idtoiso(p)$ 的形式,那么 $\gamma_a$ 将等于 $\idtoiso(p_a)$(这可以通过对 $p$ 归纳轻松证明)。
  因此,$\isotoid(\gamma_a)$ 将等于 $p_a$,所以通过函数扩展性,我们将有 $\id{\bar\gamma}{p}$,这正是我们需要的。

  最后,考虑复合
  \[(F\cong G)\to  (\id F G) \to (F\cong G). \]
  由于自然变换的同一性可以通过分量来测试,只需证明对于每个 $a$,我们有 $\id{\idtoiso(\bar\gamma)_a}{\gamma_a}$。
  但如上所述,我们有 $\id{\idtoiso(\bar\gamma)_a}{\idtoiso((\bar\gamma)_a)}$,而 $\id{(\bar\gamma)_a}{\isotoid(\gamma_a)}$ 通过函数扩展性计算得到。
  由于 $\isotoid$ 和 $\idtoiso$ 是逆函数,我们有 $\id{\idtoiso(\bar\gamma)_a}{\gamma_a}$,如所希望的那样。
\end{proof}

特别地,范畴(相对于前范畴)之间的自然同构函子是相等的。

\mentalpause

我们现在定义所有常见的函子和自然变换的合成方式。

\begin{defn}\label{ct:functor-composition}
对于函子 $F:A\to B$ 和 $G:B\to C$,它们的复合 $G\circ F:A\to C$ 定义如下:
\begin{itemize}
  \item 复合 $(G_0\circ F_0) : A_0 \to C_0$
  \item 对每个 $a,b:A$,复合
  \[(G_{Fa,Fb}\circ F_{a,b}):\hom_A(a,b) \to \hom_C(GFa,GFb)。\]
\end{itemize}
公理很容易验证。
\end{defn}

\begin{defn}\label{ct:whisker}
对于函子 $F:A\to B$ 和 $G,H:B\to C$ 以及自然变换 $\gamma:G\to H$,复合 $(\gamma F):GF\to HF$ 定义如下:
\begin{itemize}
  \item 对每个 $a:A$,分量为 $\gamma_{Fa}$。
\end{itemize}
自然性很容易验证。
同样,对于上述的 $\gamma$ 和 $K:C\to D$,复合 $(K\gamma):KG\to KH$ 定义如下:
\begin{itemize}
  \item 对每个 $b:B$,分量为 $K(\gamma_b)$。
\end{itemize}
\end{defn}

\begin{lem}\label{ct:interchange}
对于函子 $F,G:A\to B$ 和 $H,K:B\to C$ 以及自然变换 $\gamma:F\to G$ 和 $\delta:H\to K$,我们有
\[\id{(\delta G)(H\gamma)}{(K\gamma)(\delta F)}。\]
\end{lem}
\begin{proof}
  分量检查足够了:在 $a:A$ 处我们有
  \begin{align*}
    ((\delta G)(H\gamma))_a
    &\jdeq (\delta G)_{a}(H\gamma)_a\\
    &\jdeq \delta_{Ga}\circ H(\gamma_a)\\
    &= K(\gamma_a) \circ \delta_{Fa} \tag{通过 $\delta$ 的自然性}\\
    &\jdeq (K \gamma)_a\circ (\delta F)_a\\
    &\jdeq ((K \gamma)(\delta F))_a。\qedhere
  \end{align*}
\end{proof}

\index{水平合成!自然变换的 (horizontal composition of natural transformations)}%
\index{经典!范畴论 (classical category theory)}%
在经典的范畴论中,通常定义 $\gamma:F\to G$ 和 $\delta:H\to K$ 的“水平合成 (horizontal composite)”为 ${(\delta G)(H\gamma)}$ 和 ${(\gamma\delta)(\delta F)}$ 的公共值。
我们将避免这样做,因为虽然它们相等,但这两个变换并不是\emph{定义上}相等的。
这也带来了这样的结果,即我们可以明确地使用符号 $\circ$(或并列)进行所有种类的合成:只有一种方式可以合成两个自然变换(而不是在任一侧将自然变换与函子合成)。

\begin{lem}\label{ct:functor-assoc}
函子合成是结合的:$\id{H(GF)}{(HG)F}$。
\end{lem}
\begin{proof}
  由于函数的合成是结合的,这在作用于对象和同态时立刻就可以得到。
  由于同源集是集合,其余的数据是自动的。
\end{proof}

\cref{ct:functor-assoc} 中的等式同样不是定义上的。
(函数合成是定义上的结合,但进入函子的公理也必须合成,这打破了定义上的结合性。)因此,我们还需要知道关于结合性的\emph{一致性 (coherence)}。

\begin{lem}\label{ct:pentagon}
\cref{ct:functor-assoc} 是一致的,即以下五边形 (pentagon) 的等式是可交换的:
\[ \xymatrix{ & K(H(GF)) \ar@{=}[dl] \ar@{=}[dr]\\
(KH)(GF) \ar@{=}[d] && K((HG)F) \ar@{=}[d]\\
  ((KH)G)F && (K(HG))F \ar@{=}[ll].}
\]
\end{lem}
\begin{proof}
  正如在 \cref{ct:functor-assoc} 中一样,这在作用于对象时是显而易见的,其余的是自动的。
\end{proof}

我们将从此滥用符号,通过书写 $H\circ G\circ F$ 或 $HGF$ 来表示 $H(GF)$ 或 $(HG)F$,在必要时沿 \cref{ct:functor-assoc} 运输。
对于单位我们也有类似的一致性结果。

\begin{lem}\label{ct:units}
对于函子 $F:A\to B$,我们有等式 $\id{(1_B\circ F)}{F}$ 和 $\id{(F\circ 1_A)}{F}$,并且给定 $G:B\to C$ 时,以下三角形的等式是可交换的:
\[ \xymatrix{
  G\circ (1_B \circ F) \ar@{=}[rr] \ar@{=}[dr] &&
  (G\circ 1_B)\circ F \ar@{=}[dl] \\
  & G \circ F.}
\]
\end{lem}

请参阅 \cref{ct:pre2cat,ct:2cat} 以进一步发展这些思想。


\section{伴随 (Adjunctions)}
\label{sec:adjunctions}

伴随函子的定义是直接的;主要的有趣方面来源于证明相关性。

\begin{defn}\label{ct:adjoints}
一个函子 $F:A\to B$ 是一个\define{左伴随 (left adjoint)},如果存在
\begin{itemize}
  \item 一个函子 $G:B\to A$。
  \item 一个自然变换 $\eta:1_A \to GF$ (称为\define{单元 (unit)})。
  \item 一个自然变换 $\epsilon:FG\to 1_B$ (称为\define{余单元 (counit)})。
  \item $\id{(\epsilon F)(F\eta)}{1_F}$。
  \item $\id{(G\epsilon)(\eta G)}{1_G}$。
\end{itemize}
\end{defn}

最后两个等式称为\define{三角等式 (triangle identities)} 或\define{锯齿等式 (zigzag identities)}。
我们将留给读者定义右伴随 (right adjoints) 的类似定义。

\begin{lem}\label{ct:adjprop}
如果 $A$ 是一个范畴(但 $B$ 可能只是一个前范畴),那么类型“$F$ 是一个左伴随”是一个纯命题。
\end{lem}
\begin{proof}
  假设我们给定 $(G,\eta,\epsilon)$ 以及三角等式,并且也有 $(G',\eta',\epsilon')$。
  定义 $\gamma:G\to G'$ 为 $(G'\epsilon)(\eta' G)$,并且定义 $\delta:G'\to G$ 为 $(G\epsilon')(\eta G')$。
  那么
  \begin{align*}
    \delta\gamma &=
    (G\epsilon')(\eta G')(G'\epsilon)(\eta'G)\\
    &= (G\epsilon')(G F G'\epsilon)(\eta G' F G)(\eta'G)\\
    &= (G\epsilon)(G\epsilon'FG)(G F \eta' G)(\eta G)\\
    &= (G\epsilon)(\eta G)\\
    &= 1_G
  \end{align*}
  使用 \cref{ct:interchange} 和三角等式。
  类似地,我们证明 $\id{\gamma\delta}{1_{G'}}$,因此 $\gamma$ 是一个自然同构 $G\cong G'$。
  通过 \cref{ct:functor-cat},我们有一个同一性 $\id G {G'}$。

  现在我们需要知道当 $\eta$ 和 $\epsilon$ 沿此同一性运输时,它们变得等于 $\eta'$ 和 $\epsilon'$。
  通过 \cref{ct:idtoiso-trans},此运输通过与 $\gamma$ 或 $\delta$ 适当地复合来给出。
  对于 $\eta$,这产生了
  \begin{equation*}
  (G'\epsilon F)(\eta'GF)\eta
  = (G'\epsilon F)(G'F\eta)\eta'
  = \eta'
  \end{equation*}
  使用 \cref{ct:interchange} 和三角等式。
  对于 $\epsilon$ 的情况是类似的。
  最后,三角等式自动正确地运输,因为同源集是集合。
\end{proof}

在 \cref{sec:yoneda} 中我们将给出 \cref{ct:adjprop} 的另一个证明。

\section{等价 (Equivalences)}
\label{sec:equivalences}

在范畴论 (category theory) 中,通常将\emph{范畴的等价 (equivalence of categories)}定义为一个函子 $F:A\to B$,使得存在一个函子 $G:B\to A$ 以及自然同构 $F G \cong 1_B$ 和 $G F \cong 1_A$。
然而,与作为伴随 (adjunction) 的性质不同,如果不进行截断,这将不是一个纯命题,原因与拟逆元 (quasi-inverses) 的类型行为不良相同(参见 \cref{sec:quasi-inverses})。
如同在 \cref{sec:hae} 中那样,我们可以通过使用通常的\emph{伴随等价 (adjoint equivalence)}的概念来避免这个问题。
\indexdef{伴随!等价 (adjoint equivalence)!范畴 (precategories)}

\begin{defn}\label{ct:equiv}
一个函子 $F:A\to B$ 是一个\define{范畴的等价 (equivalence of (pre)categories)},
\indexdef{等价!范畴的 (equivalence of (pre)categories)}%
\indexdef{范畴!等价 (category equivalence)}%
\indexdef{前范畴!等价 (precategory equivalence)}%
\index{函子!等价 (functor equivalence)}%
如果它是一个左伴随,并且其 $\eta$ 和 $\epsilon$ 是同构。
我们记 $\cteqv A B$ 为从 $A$ 到 $B$ 的范畴等价的类型。
\end{defn}

根据 \cref{ct:adjprop,ct:isoprop},如果 $A$ 是一个范畴,那么“$F$ 是前范畴的等价”这个类型是一个纯命题。

\begin{lem}\label{ct:adjointification}
如果对于 $F:A\to B$ 存在 $G:B\to A$ 和同构 $GF\cong 1_A$ 和 $FG\cong 1_B$,那么 $F$ 是前范畴的等价。
\end{lem}
\begin{proof}
  类似于 \cref{thm:equiv-iso-adj} 中关于类型等价的证明。
\end{proof}

\begin{defn}\label{ct:full-faithful}
我们说一个函子 $F:A\to B$ 是\define{忠实的 (faithful)},
\indexdef{函子!忠实 (faithful)}%
\index{忠实函子 (faithful functor)}%
如果对于所有 $a,b:A$,函数
\[F_{a,b}:\hom_A(a,b) \to \hom_B(Fa,Fb)\]
是单射 (injective) 的,并且\define{满的 (full)},
\indexdef{函子!满 (full)}%
\indexdef{满函子 (full functor)}%
如果对于所有 $a,b:A$ 这个函数是满射 (surjective) 的。
如果它同时是两者(因此每个 $F_{a,b}$ 是一个等价),我们说 $F$ 是\define{完全忠实的 (fully faithful)}。
\indexdef{函子!完全忠实 (fully faithful)}%
\indexdef{完全忠实函子 (fully faithful functor)}%
\end{defn}

\begin{defn}\label{ct:split-essentially-surjective}
我们说一个函子 $F:A\to B$ 是\define{分裂本质满射的 (split essentially surjective)},
\indexdef{函子!分裂本质满射 (split essentially surjective)}%
\indexdef{分裂!本质满射函子 (split essentially surjective functor)}%
如果对于所有 $b:B$,存在一个 $a:A$,使得 $Fa\cong b$。
\end{defn}

\begin{lem}\label{ct:ffeso}
对于任意前范畴 $A$ 和 $B$ 以及函子 $F:A\to B$,以下类型是等价的。
\begin{enumerate}
  \item $F$ 是前范畴的等价。\label{item:ct:ffeso1}
  \item $F$ 是完全忠实并且分裂本质满射的。\label{item:ct:ffeso2}
\end{enumerate}
\end{lem}
\begin{proof}
  假设 $F$ 是前范畴的等价,并指定了 $G,\eta,\epsilon$。
  那么我们有函数
  \begin{align*}
    \hom_B(Fa,Fb) &\to \hom_A(a,b),\\
    g &\mapsto \inv{\eta_b}\circ G(g)\circ \eta_a。
  \end{align*}
  对于 $f:\hom_A(a,b)$,我们有
  \[ \inv{\eta_{b}}\circ G(F(f))\circ \eta_{a}  =
  \inv{\eta_{b}} \circ \eta_{b} \circ f=
  f
  \]
  而对于 $g:\hom_B(Fa,Fb)$,我们有
  \begin{align*}
    F(\inv{\eta_b} \circ G(g)\circ\eta_a)
    &= F(\inv{\eta_b})\circ F(G(g))\circ F(\eta_a)\\
    &= \epsilon_{Fb}\circ F(G(g))\circ F(\eta_a)\\
    &= g\circ\epsilon_{Fa}\circ F(\eta_a)\\
    &= g
  \end{align*}
  使用 $\epsilon$ 的自然性 (naturality),以及三角等式两次。
  因此,$F_{a,b}$ 是一个等价,因此 $F$ 是完全忠实的。
  最后,对于任意 $b:B$,我们有 $Gb:A$ 和 $\epsilon_b:FGb\cong b$。

  另一方面,假设 $F$ 是完全忠实并且分裂本质满射的。
  定义 $G_0:B_0\to A_0$ 通过将 $b:B$ 发送到指定的本质分裂给出的 $a:A$,并写 $\epsilon_b$ 为同样指定的同构 $FGb\cong b$。

  现在对于任意 $g:\hom_B(b,b')$,定义 $G(g):\hom_A(Gb,Gb')$ 为唯一的态射,使得 $\id{F(G(g))}{\inv{(\epsilon_{b'})}\circ g \circ \epsilon_b }$(由于 $F$ 是完全忠实的,该态射存在)。
  最后,对于 $a:A$ 定义 $\eta_a:\hom_A(a,GFa)$ 为唯一的态射,使得 $\id{F\eta_a}{\inv{\epsilon_{Fa}}}$。
  很容易验证 $G$ 是一个函子,并且 $(G,\eta,\epsilon)$ 展示了 $F$ 作为一个前范畴的等价。

  现在考虑复合~\ref{item:ct:ffeso1}$\to$\ref{item:ct:ffeso2}$\to$\ref{item:ct:ffeso1}。
  我们显然恢复了相同的函数 $G_0:B_0 \to A_0$。
  对于同态集上的 $G$ 的作用,我们必须证明对于 $g:\hom_B(b,b')$,$G(g)$ 是(必然唯一的)态射,使得 $F(G(g)) = \inv{(\epsilon_{b'})}\circ g \circ \epsilon_b$。
  但这个方程通过假设的 $\epsilon$ 的自然性成立。
  我们同样显然恢复了 $\epsilon$,而 $\eta$ 是通过 $\id{F\eta_a}{\inv{\epsilon_{Fa}}}$(这是伴随等价结构中假定的三角等式之一)唯一确定的。
  因此,这个复合等于恒等式。

  最后,考虑另一个复合~\ref{item:ct:ffeso2}$\to$\ref{item:ct:ffeso1}$\to$\ref{item:ct:ffeso2}。
  由于完全忠实是一个纯命题,足以观察到我们为每个 $b:B$ 恢复了相同的 $a:A$ 和同构 $F a \cong b$。
  但这是显然的,因为我们使用了这个函数和同构来定义 $G_0$ 和 $\epsilon$ 在~\ref{item:ct:ffeso1} 中,这反过来又正是我们用来再次恢复~\ref{item:ct:ffeso2} 的内容。
  因此,两个方向上的复合都等于恒等式,因此我们有一个等价 $\eqv{\text{\ref{item:ct:ffeso1}}}{\text{\ref{item:ct:ffeso2}}}$。
\end{proof}

然而,如果 $A$ 不是一个范畴,那么 \cref{ct:ffeso} 中的任意类型可能不一定是一个纯命题。
这提示我们也考虑以下概念。

\begin{defn}\label{ct:essentially-surjective}
一个函子 $F:A\to B$ 是\define{本质满射 (essentially surjective)}的,
\indexdef{函子!本质满射 (essentially surjective)}%
\indexdef{本质满射函子 (essentially surjective functor)}%
如果对于所有 $b:B$,\emph{仅仅}存在一个 $a:A$ 使得 $Fa\cong b$。
我们说 $F$ 是一个\define{弱等价 (weak equivalence)},
\indexsee{等价!弱等价 (weak equivalence)的 (equivalence of (pre)categories)}{weak equivalence}%
\indexdef{弱等价!前范畴 (weak equivalence of precategories)}%
\indexsee{函子!弱等价 (weak equivalence)的 (functor weak equivalence)}{weak equivalence}%
如果它是完全忠实并且本质满射的。
\end{defn}

作为一个弱等价始终是一个纯命题。
然而,对于范畴之间的函子,等价与弱等价没有区别。

\index{接受度 (acceptance)}
\begin{lem}\label{ct:catweq}
如果 $F:A\to B$ 是完全忠实的并且 $A$ 是一个范畴,那么对于任何 $b:B$ 类型 $\sm{a:A} (Fa\cong b)$ 是一个纯命题。
因此,一个范畴之间的函子如果且仅如果它是一个弱等价,则是一个等价。
\end{lem}
\begin{proof}
  假设给定 $(a,f)$ 和 $(a',f')$ 在 $\sm{a:A} (Fa\cong b)$ 中。
  然后 $\inv{f'}\circ f$ 是一个同构 $Fa \cong Fa'$。
  由于 $F$ 是完全忠实的,我们有 $g:a\cong a'$,其中 $Fg = \inv{f'}\circ f$。
  由于 $A$ 是一个范畴,我们有 $p:a=a'$,其中 $\idtoiso(p)=g$。
  现在 $Fg = \inv{f'}\circ f$ 意味着 $\trans{(\map{(F_0)}{p})}{f} = f'$,因此(根据依赖对类型中相等性的特征)$(a,f)=(a',f')$。

  因此,对于域为范畴的完全忠实函子,本质满射等价于分裂本质满射,因此,弱等价等价于等价。
\end{proof}

这是我们范畴论相对于基于集合的方法的一个重要优势。
在一个纯集合论定义的范畴中,“每个完全忠实并且本质满射的函子都是一个范畴等价”这一命题等价于选择公理 \choice{}。
在这里,我们免费获得了它,作为选择唯一性原则 (principle of unique choice) 的范畴论版本 (\cref{sec:unique-choice})。
(实际上,这个性质刻画了前范畴中的范畴;参见 \cref{sec:rezk}。)

另一方面,以下对范畴等价的刻画可能更有用。

\begin{defn}\label{ct:isocat}
一个函子 $F:A\to B$ 是一个\define{范畴的同构 (isomorphism of (pre)categories)},
\indexdef{同构!范畴的 (isomorphism of (pre)categories)}%
\indexdef{范畴!同构 (category isomorphism)}%
\indexdef{前范畴!同构 (precategory isomorphism)}%
如果 $F$ 是完全忠实的并且 $F_0:A_0\to B_0$ 是一个类型等价。
\end{defn}

这个定义是我们的一般规则的一个例外(参见 \cref{sec:basics-equivalences}),即只在集合和类集合样的对象上使用“同构”一词。
然而,在这里它确实带有适当的含义,因为对于一般的前范畴,同构比等价更强。

注意,作为前范畴的同构始终是一个纯命题。
我们记 $A\cong B$ 为从 $A$ 到 $B$ 的前范畴同构的类型。

\begin{lem}\label{ct:isoprecat}
对于前范畴 $A$ 和 $B$ 以及函子 $F:A\to B$,以下命题是等价的。
\begin{enumerate}
  \item $F$ 是前范畴的同构。\label{item:ct:ipc1}
  \item 存在 $G:B\to A$ 和 $\eta:1_A = GF$ 以及 $\epsilon:FG=1_B$,使得\label{item:ct:ipc2}
  \begin{equation}
    \apfunc{(\lam{H} F H)}({\eta}) = \apfunc{(\lam{K} K F)}({\opp\epsilon})。\label{eq:ct:isoprecattri}
  \end{equation}
  \item 仅仅存在 $G:B\to A$ 和 $\eta:1_A = GF$ 以及 $\epsilon:FG=1_B$。\label{item:ct:ipc3}
\end{enumerate}
\end{lem}

注意,如果 $B_0$ 不是一个 1-类型,那么~\eqref{eq:ct:isoprecattri} 可能不是一个纯命题。

\begin{proof}
  首先注意,由于同源集是集合,函子之间相等性的相等由其对象部分唯一确定。
  因此,根据函数外延性 (function extensionality),~\eqref{eq:ct:isoprecattri} 等价于
  \begin{equation}
    \map{(F_0)}{\eta_0}_a = \opp{(\epsilon_0)}_{F_0 a}。\label{eq:ct:ipctri}
  \end{equation}
  对于所有 $a:A_0$。
  注意这正是 $G_0$,$\eta_0$ 和 $\epsilon_0$ 作为证明 $F_0$ 是一个半伴随等价 (half adjoint equivalence) 的三角等式。

  现在假设~\ref{item:ct:ipc1}。
  令 $G_0:B_0 \to A_0$ 为 $F_0$ 的逆元,具有 $\eta_0: \idfunc[A_0] = G_0 F_0$ 和 $\epsilon_0:F_0G_0 = \idfunc[B_0]$ 满足三角等式,这正是~\eqref{eq:ct:ipctri}。
  现在定义 $G_{b,b'}:\hom_B(b,b') \to \hom_A(G_0b,G_0b')$ 为
  \[ G_{b,b'}(g) \defeq
  \inv{(F_{G_0b,G_0b'})}\Big(\idtoiso(\opp{(\epsilon_0)}_{b'}) \circ g \circ \idtoiso((\epsilon_0)_b)\Big)
  \]
  (使用假设 $F$ 是完全忠实的)。
  由于 \idtoiso 取逆元到逆元,且复合到组合,并且 $F$ 是一个函子,因此 $G$ 是一个函子。

  按定义,我们有 $(GF)_0 \jdeq G_0 F_0$,通过 $\eta_0$ 等于 $\idfunc[A_0]$。
  为了获得 $1_A = GF$,我们需要证明当沿着 $\eta_0$ 运输时,同源集的恒等函数变得等于复合 $G_{Fa,Fa'} \circ F_{a,a'}$。
  换句话说,对于任意 $f:\hom_A(a,a')$,我们必须有
  \begin{multline*}
    \idtoiso((\eta_0)_{a'}) \circ f \circ \idtoiso(\opp{(\eta_0)}_a)\\
    = \inv{(F_{GFa,GFa'})}\Big(\idtoiso(\opp{(\epsilon_0)}_{Fa'})
    \circ F_{a,a'}(f) \circ \idtoiso((\epsilon_0)_{Fa})\Big)。
  \end{multline*}
  但这等价于
  \begin{multline*}
  (F_{GFa,GFa'})\Big(\idtoiso((\eta_0)_{a'}) \circ f \circ \idtoiso(\opp{(\eta_0)}_a)\Big)\\
  = \idtoiso(\opp{(\epsilon_0)}_{Fa'})
  \circ F_{a,a'}(f)
  \circ \idtoiso((\epsilon_0)_{Fa})。
  \end{multline*}
  这跟随于 $F$ 的函子性,$F$ 保持 \idtoiso 的事实,以及~\eqref{eq:ct:ipctri}。
  因此我们有 $\eta:1_A = GF$。

  另一方面,我们有 $(FG)_0\jdeq F_0 G_0$,通过 $\epsilon_0$ 等于 $\idfunc[B_0]$。
  为了获得 $FG=1_B$,我们需要证明当沿着 $\epsilon_0$ 运输时,同源集的恒等函数变得等于复合 $F_{Gb,Gb'} \circ G_{b,b'}$。
  也就是说,对于任意 $g:\hom_B(b,b')$,我们必须有
  \begin{multline*}
    F_{Gb,Gb'}\Big(\inv{(F_{Gb,Gb'})}\Big(\idtoiso(\opp{(\epsilon_0)}_{b'}) \circ g \circ \idtoiso((\epsilon_0)_b)\Big)\Big)\\
    = \idtoiso((\opp{\epsilon_0})_{b'}) \circ g \circ \idtoiso((\epsilon_0)_b)。
  \end{multline*}
  但这仅仅是 $\inv{(F_{Gb,Gb'})}$ 是 $F_{Gb,Gb'}$ 的逆元的事实。
  我们已经提到~\eqref{eq:ct:isoprecattri} 等价于~\eqref{eq:ct:ipctri},所以~\ref{item:ct:ipc2} 成立。

  反过来,假设~\ref{item:ct:ipc2};那么 $G$ 的对象部分,$\eta$ 和 $\epsilon$ 以及~\eqref{eq:ct:ipctri} 的对象部分表明 $F_0$ 是一个类型等价。
  而对于 $a,a':A_0$,我们定义 $\overline{G}_{a,a'}: \hom_B(Fa,Fa') \to \hom_A(a,a')$ 为
  \begin{equation}
    \overline{G}_{a,a'}(g) \defeq \idtoiso(\opp{\eta})_{a'} \circ G(g) \circ \idtoiso(\eta)_a。\label{eq:ct:gbar}
  \end{equation}
  根据 $\idtoiso(\eta)$ 的自然性,对于任意 $f:\hom_A(a,a')$,我们有
  \begin{align*}
    \overline{G}_{a,a'}(F_{a,a'}(f))
    &= \idtoiso(\opp{\eta})_{a'} \circ G(F(f)) \circ \idtoiso(\eta)_a\\
    &= \idtoiso(\opp{\eta})_{a'} \circ \idtoiso(\eta)_{a'} \circ f \\
    &= f。
  \end{align*}
  另一方面,对于 $g:\hom_B(Fa,Fa')$,我们有
  \begin{align*}
    F_{a,a'}(\overline{G}_{a,a'}(g))
    &= F(\idtoiso(\opp{\eta})_{a'}) \circ F(G(g)) \circ F(\idtoiso(\eta)_a)\\
    &= \idtoiso(\epsilon)_{Fa'}
    \circ F(G(g))
    \circ \idtoiso(\opp{\epsilon})_{Fa}\\
    &= \idtoiso(\epsilon)_{Fa'}
    \circ \idtoiso(\opp{\epsilon})_{Fa'}
    \circ g\\
    &= g。
  \end{align*}
  (这里需要一些关于 \idtoiso 和 whiskering 兼容性的引理,我们留给读者去陈述和证明。)
  因此,$F_{a,a'}$ 是一个等价,所以 $F$ 是完全忠实的;即~\ref{item:ct:ipc1} 成立。

  现在复合~\ref{item:ct:ipc1}$\to$\ref{item:ct:ipc2}$\to$\ref{item:ct:ipc1} 等于恒等式,因为~\ref{item:ct:ipc1} 是一个纯命题。
  另一方面,跟踪以上构造,我们看到复合~\ref{item:ct:ipc2}$\to$\ref{item:ct:ipc1}$\to$\ref{item:ct:ipc2} 本质上保留了 $G_0$,$\eta_0$,$\epsilon_0$ 的对象部分以及~\eqref{eq:ct:isoprecattri} 的对象部分。
  而在后面三种情况下,对象部分就是全部,因为同源集是集合。

  因此,足以证明我们恢复了 $G$ 在同源集上的作用。
  换句话说,我们必须证明如果 $g:\hom_B(b,b')$,那么
  \[ G_{b,b'}(g) =
  \overline{G}_{G_0b,G_0b'}\Big(\idtoiso(\opp{(\epsilon_0)}_{b'}) \circ g \circ \idtoiso((\epsilon_0)_b)\Big)
  \]
  其中 $\overline{G}$ 定义为~\eqref{eq:ct:gbar}。
  然而,这跟随于 $G$ 的函子性和\emph{另一}三角等式,我们在 \cref{cha:equivalences} 中看到这等价于~\eqref{eq:ct:ipctri}。

  现在,由于~\ref{item:ct:ipc1} 是一个纯命题,~\ref{item:ct:ipc2} 也是如此,所以足以证明它们逻辑上等价于~\ref{item:ct:ipc3}。
  当然,~\ref{item:ct:ipc2}$\to$\ref{item:ct:ipc3},所以让我们假设~\ref{item:ct:ipc3}。
  由于~\ref{item:ct:ipc1} 是一个纯命题,我们可以假设给定 $G$,$\eta$ 和 $\epsilon$。
  然后 $G_0$ 连同 $\eta$ 和 $\epsilon$ 表明 $F_0$ 是一个等价。
  此外,我们还有自然同构 $\idtoiso(\eta):1_A\cong GF$ 和 $\idtoiso(\epsilon):FG\cong 1_B$,因此根据 \cref{ct:adjointification},$F$ 是一个前范畴的等价,特别是完全忠实的。
\end{proof}

从 \cref{ct:isoprecat}\ref{item:ct:ipc2} 和函子范畴中的 $\idtoiso$,我们可以立即得出任何前范畴的同构都是等价的。
对于前范畴,反过来可能会失败。

\begin{eg}\label{ct:chaotic}
设 $X$ 是一个类型且 $x_0:X$ 是一个元素,并令 $X_{\mathrm{ch}}$ 表示 $X$ 上的\emph{混乱 (chaotic)}\indexdef{混乱前范畴 (chaotic precategory)}或\emph{不分离 (indiscrete)}\indexdef{不分离前范畴 (indiscrete precategory)}前范畴。
按照定义,我们有 $(X_{\mathrm{ch}})_0\defeq X$,并且对于所有 $x,x'$,$\hom_{X_{\mathrm{ch}}}(x,x') \defeq \unit$。
然后唯一的函子 $X_{\mathrm{ch}}\to \unit$ 是一个前范畴的等价,但除非 $X$ 是收缩的 (contractible),否则它不是同构。

这个例子也表明,一个前范畴可以等价于一个范畴而不自身是一个范畴。
当然,如果一个前范畴\emph{是同构的}到一个范畴,那么它自身必须是一个范畴。
\end{eg}

然而,对于范畴,这两个概念是一致的。

\begin{lem}\label{ct:eqv-levelwise}
对于范畴 $A$ 和 $B$,一个函子 $F:A\to B$ 如果且仅如果它是范畴的同构,则是范畴的等价。
\end{lem}
\begin{proof}
  由于两者都是纯命题,所以足以证明它们逻辑上等价。
  因此,首先假设 $F$ 是范畴的等价,并给定了 $(G,\eta,\epsilon)$。
  我们已经看到 $F$ 是完全忠实的。
  根据 \cref{ct:functor-cat},自然同构 $\eta$ 和 $\epsilon$ 产生了恒等式 $\id{1_A}{GF}$ 和 $\id{FG}{1_B}$,因此特别是 $\id{\idfunc[A]}{G_0\circ F_0}$ 和 $\id{F_0\circ G_0}{\idfunc[B]}$ 的恒等式。
  因此,$F_0$ 是一个类型等价。

  反过来,假设 $F$ 是完全忠实的并且 $F_0$ 是一个类型等价,假设其逆元是 $G_0$。
  那么对于每个 $b:B$,我们有 $G_0 b:A$ 并且有一个恒等式 $\id{FGb}{b}$,因此有一个同构 $FGb\cong b$。
  因此,根据 \cref{ct:ffeso},$F$ 是一个范畴的等价。
\end{proof}

当然,还有第三个相似的概念用于(前)范畴:相等性。
然而,一致性公理 (univalence axiom) 表明它与同构一致。

\begin{lem}\label{ct:cat-eq-iso}
如果 $A$ 和 $B$ 是前范畴,那么函数
\[(\id A B) \to (A\cong B)\]
(通过从恒等函子归纳定义)是一个类型等价。
\end{lem}
\begin{proof}
  正如通常对于依赖和类型,给出一个 $\id A B$ 的元素等价于给出
  \begin{itemize}
    \item 一个恒等式 $P_0:\id{A_0}{B_0}$,
    \item 对于每个 $a,b:A_0$,一个恒等式
    \[P_{a,b}:\id{\hom_A(a,b)}{\hom_B(\trans {P_0} a,\trans {P_0} b)},\]
    \item 恒等式 $\id{\trans {(P_{a,a})} {1_a}}{1_{\trans {P_0} a}}$ 和
    \narrowequation{\id{\trans {(P_{a,c})} {gf}}{\trans {(P_{b,c})} g \circ \trans {(P_{a,b})} f}。}
  \end{itemize}
  (再次使用同源集的身份类型是纯命题的事实。)
  然而,根据一致性,这等价于给出
  \begin{itemize}
    \item 一个类型的等价 $F_0:\eqv{A_0}{B_0}$,
    \item 对于每个 $a,b:A_0$,一个类型的等价
    \[F_{a,b}:\eqv{\hom_A(a,b)}{\hom_B(F_0 (a),F_0 (b))},\]
    \item 和恒等式 $\id{F_{a,a}(1_a)}{1_{F_0 (a)}}$ 和 $\id{F_{a,c}(gf)}{F_{b,c} (g)\circ F_{a,b} (f)}$。
  \end{itemize}
  但这正是一个函子 $F:A\to B$,它是一个范畴的同构。
  并且通过对恒等式的归纳,这个等价 $\eqv{(\id A B)}{(A\cong B)}$ 等于通过归纳获得的那个。
\end{proof}

因此,对于范畴,相等性也等同于等价。
我们可以将此解释为,范畴、函子和自然变换不仅形成了一个前 2-范畴 (pre-2-category),还形成了一个 2-范畴(参见 \cref{ct:pre2cat})。

\begin{thm}\label{ct:cat-2cat}
如果 $A$ 和 $B$ 是范畴,那么函数
\[(\id A B) \to (\cteqv A B)\]
(通过从恒等函子归纳定义)是一个类型等价。
\end{thm}
\begin{proof}
  根据 \cref{ct:cat-eq-iso,ct:eqv-levelwise}。
\end{proof}

因此,范畴的类型是一个 2-类型。
因为 $\cteqv A B$ 是函子类型的一个子类型,而函子是范畴的对象,所以它是一个 1-类型;因此 $\id A B$ 的恒等类型也是 1-类型。

\section{Yoneda 引理 (The Yoneda Lemma)}
\label{sec:yoneda}
\index{Yoneda!lemma|(}

回顾一下,我们有一个范畴 \uset,其对象是集合,其态射是函数。现在我们展示每个前范畴都有一个取值为 \uset 的 $\hom$-函子。首先我们需要定义(前)范畴的对偶与积。

\begin{defn}\label{ct:opposite-category}
对于一个前范畴 $A$,其 \define{对偶 (opposite)}
\indexdef{opposite of a (pre)category}{(前)范畴的对偶}%
\indexdef{precategory!opposite}{前范畴的对偶}%
\indexdef{category!opposite}{范畴的对偶}%
$A\op$ 是一个具有相同对象类型的前范畴,其中 $\hom_{A\op}(a,b) \defeq \hom_A(b,a)$,而其恒等元和复合则继承自 $A$。
\end{defn}

\begin{defn}\label{ct:prod-cat}
对于前范畴 $A$ 和 $B$,它们的 \define{积 (product)}
\index{precategory!product of}{前范畴的积}%
\index{category!product of}{范畴的积}%
\index{product!of (pre)categories}{(前)范畴的积}%
$A\times B$ 是一个前范畴,其 $(A\times B)_0 \defeq A_0 \times B_0$ 且
\[\hom_{A\times B}((a,b),(a',b')) \defeq \hom_A(a,a') \times \hom_B(b,b')。\]
恒等元定义为 $1_{(a,b)}\defeq (1_a,1_b)$,复合定义为
\narrowequation{(g,g')(f,f') \defeq ((gf),(g'f'))。}
\end{defn}

\begin{lem}\label{ct:functorexpadj}
对于前范畴 $A,B,C$,下列类型是等价的。
\begin{enumerate}
  \item 函子 $A\times B\to C$。
  \item 函子 $A\to C^B$。
\end{enumerate}
\end{lem}
\begin{proof}
  给定 $F:A\times B\to C$,对于任意 $a:A$,我们显然有一个函子 $F_a : B\to C$。这给出了一个函数 $A_0 \to (C^B)_0$。接下来,对于任意 $f:\hom_A(a,a')$,我们对于任意 $b:B$ 都有态射 $F_{(a,b),(a',b)}(f,1_b):F_a(b) \to F_{a'}(b)$。这些是自然变换 $F_a \to F_{a'}$ 的分量。在 $a$ 上的函子性易于验证,因此我们有一个函子 $\hat{F}:A\to C^B$。

  反过来,假设给定 $G:A\to C^B$。那么对于任意 $a:A$ 和 $b:B$,我们有对象 $G(a)(b):C$,给出一个函数 $A_0 \times B_0 \to C_0$。对于 $f:\hom_A(a,a')$ 和 $g:\hom_B(b,b')$,我们有态射
  \begin{equation*}
    G(a')_{b,b'}(g)\circ G_{a,a'}(f)_b = G_{a,a'}(f)_{b'} \circ  G(a)_{b,b'}(g)
  \end{equation*}
  在 $\hom_C(G(a)(b), G(a')(b'))$ 中。函子性再次容易验证,因此我们有一个函子 $\check{G}:A\times B \to C$。

  最后,这些操作是互逆的也很明显。
\end{proof}

现在对于任何前范畴 $A$,我们有一个 $\hom$-函子
\indexdef{hom-functor}{$\hom$-函子}%
\[\hom_A : A\op \times A \to \uset。\]
它将一个对 $(a,b): (A\op)_0 \times A_0 \jdeq A_0 \times A_0$ 映射为集合 $\hom_A(a,b)$。对于一个态射 $(f,f') : \hom_{A\op\times A}((a,b),(a',b'))$,根据定义我们有 $f:\hom_A(a',a)$ 和 $f':\hom_A(b,b')$,因此我们可以定义
\begin{align*}
(\hom_A)_{(a,b),(a',b')}(f,f')
&\defeq (g \mapsto (f'gf))\\
&: \hom_A(a,b) \to \hom_A(a',b')。
\end{align*}
函子性易于验证。

因此根据 \cref{ct:functorexpadj},我们有一个诱导的函子 $\y:A\to \uset^{A\op}$,我们称之为 \define{Yoneda 嵌入 (Yoneda embedding)}。
\indexdef{Yoneda!embedding}{Yoneda 嵌入}%
\indexdef{embedding!Yoneda}{嵌入!Yoneda}%

\begin{thm}[Yoneda 引理]\label{ct:yoneda}
\indexdef{Yoneda!lemma}{Yoneda 引理}
对于任何前范畴 $A$,任意 $a:A$ 和任意函子 $F:\uset^{A\op}$,我们有一个同构
\begin{equation}\label{eq:yoneda}
\hom_{\uset^{A\op}}(\y a, F) \cong Fa。
\end{equation}
此外,这在 $a$ 和 $F$ 中都是自然的。
\end{thm}
\begin{proof}
  给定一个自然变换 $\alpha:\y a \to F$,我们可以考虑分量 $\alpha_a : \y a(a) \to F a$。由于 $\y a(a)\jdeq \hom_A(a,a)$,我们有 $1_a : \y a(a)$,因此 $\alpha_a(1_a) : F a$。这给出了~\eqref{eq:yoneda} 中从左到右的一个函数 $(\alpha \mapsto \alpha_a(1_a))$。

  反过来,给定 $x:F a$,我们通过定义 $\alpha:\y a \to F$
  \[\alpha_{a'}(f) \defeq F_{a,a'}(f)(x)。\]
  自然性易于验证,因此这给出了~\eqref{eq:yoneda} 中从右到左的一个函数。

  为了证明这些是互逆的,首先假设给定 $x:F a$。那么按照上面的方法定义的 $\alpha$ 满足 $\alpha_a(1_a) = F_{a,a}(1_a)(x) = 1_{F a}(x) = x$。另一方面,如果假设给定 $\alpha:\y a \to F$ 并按照上面的方法定义 $x$,那么对于任意 $f:\hom_A(a',a)$ 我们有
  \begin{align*}
    \alpha_{a'}(f)
    &= \alpha_{a'} (\y a_{a,a'}(f)(1_a))\\
    &= (\alpha_{a'}\circ \y a_{a,a'}(f))(1_a)\\
    &= (F_{a,a'}(f)\circ \alpha_a)(1_a)\\
    &= F_{a,a'}(f)(\alpha_a(1_a))\\
    &= F_{a,a'}(f)(x)。
  \end{align*}
  因此,两个复合都等于恒等式。我们将自然性的证明留给读者。
\end{proof}

\begin{cor}\label{ct:yoneda-embedding}
Yoneda 嵌入 $\y :A\to \uset^{A\op}$ 是完全忠实的。
\end{cor}
\begin{proof}
  根据 \cref{ct:yoneda},我们有
  \[ \hom_{\uset^{A\op}}(\y a, \y b) \cong \y b(a) \jdeq \hom_A(a,b)。\]
  验证这个同构实际上是 \y 在 $\hom$ 集上的作用是容易的。
\end{proof}

\begin{cor}\label{ct:yoneda-mono}
如果 $A$ 是一个范畴,那么 $\y_0 : A_0 \to (\uset^{A\op})_0$ 是一个嵌入。特别地,如果 $\y a = \y b$,那么 $a=b$。
\end{cor}
\begin{proof}
  根据 \cref{ct:yoneda-embedding},\y 在同构集上诱导了一个同构。由于 $A$ 和 $\uset^{A\op}$ 是范畴且 \y 是一个函子,这等价于在身份类型上的同构,这就是嵌入的定义。
\end{proof}

\begin{defn}\label{ct:representable}
一个函子 $F:\uset^{A\op}$ 被称为 \define{可表示的 (representable)}
\indexdef{functor!representable}{可表示函子}%
\indexdef{representable functor}{可表示函子}%
如果存在 $a:A$ 并且有一个同构 $\y a \cong F$。
\end{defn}

\begin{thm}\label{ct:representable-prop}
如果 $A$ 是一个范畴,那么“$F$ 是可表示的”类型是一个纯命题。
\end{thm}
\begin{proof}
  根据定义“$F$ 是可表示的”恰好是 $\y_0$ 在 $F$ 上的纤维。由于 \cref{ct:yoneda-mono} 中的 $\y_0$ 是一个嵌入,因此这个纤维是一个纯命题。
\end{proof}

特别地,在一个范畴中,同一个函子的任何两个表示都是相等的。我们可以利用这一点给出 \cref{ct:adjprop} 的另一种证明。首先我们给出一种用可表示性描述伴随关系的特征。

\begin{lem}\label{ct:adj-repr}
对于任意前范畴 $A$ 和 $B$ 以及一个函子 $F:A\to B$,下列类型是等价的。
\begin{enumerate}
  \item $F$ 是一个左伴随 (left adjoint)\index{adjoint!functor}{伴随函子}。\label{item:ct:ar1}
  \item 对于每个 $b:B$,从 $A\op$ 到 \uset 的函子 $(a \mapsto \hom_B(Fa,b))$ 是可表示的\index{representable functor}{可表示函子}。\label{item:ct:ar2}
\end{enumerate}
\end{lem}
\begin{proof}
  类型~\ref{item:ct:ar2} 的一个元素由一个函数 $G_0:B_0 \to A_0$ 以及对于每个 $a:A$ 和 $b:B$ 的一个同构
  \[ \gamma_{a,b}:\hom_B(Fa,b) \cong \hom_A(a,G_0 b) \]
  构成,满足 $\gamma_{a,b}(g \circ Ff) = \gamma_{a',b}(g)\circ f$ 对于 $f:\hom_{A}(a,a')$。

  有了这个,对于 $a:A$,我们定义 $\eta_a \defeq \gamma_{a,Fa}(1_{Fa})$,对于 $b:B$,我们定义 $\epsilon_b \defeq \inv{(\gamma_{Gb,b})}(1_{Gb})$。现在对于 $g:\hom_B(b,b')$ 我们定义
  \[ G_{b,b'}(g) \defeq \gamma_{G b, b'}(g \circ \epsilon_b) \]
  验证 $G$ 是一个函子且 $\eta$ 和 $\epsilon$ 是满足三角恒等式的自然变换的过程与经典情形完全相同,并且由于它们都是纯命题,我们不会关心它们的具体值。因此,我们有一个从~\ref{item:ct:ar2} 到~\ref{item:ct:ar1} 的函数。

  反过来,如果 $F$ 是一个左伴随,我们当然有 $G_0$ 指定的,我们可以将 $\gamma_{a,b}$ 取为复合
  \[ \hom_B(Fa,b)
  \xrightarrow{G_{Fa,b}} \hom_A(GFa,Gb)
  \xrightarrow{(\blank\circ \eta_a)} \hom_A(a,Gb)。
  \]
  由于 $\eta$ 是自然的,因此这显然是自然的,并且它有一个逆元给出
  \[ \hom_A(a,Gb)
  \xrightarrow{F_{a,Gb}} \hom_B(Fa,FGb)
  \xrightarrow{(\epsilon_b \circ \blank )} \hom_A(Fa,b)
  \]
  (通过三角恒等式)。
  因此我们也有~\ref{item:ct:ar1} 到~\ref{item:ct:ar2}。

  对于复合~\ref{item:ct:ar2} 到~\ref{item:ct:ar1} 再到~\ref{item:ct:ar2},显然函数 $G_0$ 是保持的,因此只需检查我们是否得到了 $\gamma$。但是新定义的 $\gamma$ 被定义为将 $f:\hom_B(Fa,b)$ 映射为
  \begin{align*}
    G(f) \circ \eta_a
    &\jdeq \gamma_{G Fa, b}(f \circ \epsilon_{Fa}) \circ \eta_a\\
    &= \gamma_{G Fa, b}(f \circ \epsilon_{Fa} \circ F\eta_a)\\
    &= \gamma_{G Fa, b}(f)
  \end{align*}
  所以它与旧的 $\gamma$ 一致。

  最后,对于~\ref{item:ct:ar1} 到~\ref{item:ct:ar2} 再到~\ref{item:ct:ar1},我们肯定会得到对象上的函子 $G$。新的 $G_{b,b'}:\hom_B(b,b') \to \hom_A(Gb,Gb')$ 被定义为将 $g$ 映射为
  \begin{align*}
    \gamma_{G b, b'}(g \circ \epsilon_b)
    &\jdeq G(g \circ \epsilon_b) \circ \eta_{Gb}\\
    &= G(g) \circ G\epsilon_b \circ \eta_{Gb}\\
    &= G(g)
  \end{align*}
  所以它与旧的 $G$ 一致。新的 $\eta_a$ 被定义为 $\gamma_{a,Fa}(1_{Fa}) \jdeq G(1_{Fa}) \circ \eta_a$,因此它等于旧的 $\eta_a$。最后,新的 $\epsilon_b$ 被定义为 $\inv{(\gamma_{Gb,b})}(1_{Gb}) \jdeq \epsilon_b \circ F(1_{Gb})$,这也等于旧的 $\epsilon_b$。
\end{proof}

\begin{cor}\label{ct:adjprop2}[\cref{ct:adjprop}]
如果 $A$ 是一个范畴且 $F:A\to B$,那么类型“$F$ 是一个左伴随”是一个纯命题。
\end{cor}
\begin{proof}
  根据 \cref{ct:representable-prop},如果 $A$ 是一个范畴,那么 \cref{ct:adj-repr} 中的类型 \ref{item:ct:ar2} 是一个纯命题。
\end{proof}
\index{Yoneda!lemma|)}

\section{严格范畴 (Strict categories)}
\label{sec:strict-categories}

\index{bargaining|(}%

\begin{defn}\label{ct:strict-category}
一个\define{严格范畴 (strict category)}
\indexdef{category!strict}%
\indexdef{strict!category}%
是一个其对象类型为集合的预范畴。
\end{defn}

根据数学中“红鲱鱼原则 (red herring principle)”\index{red herring principle},严格范畴不一定是一个范畴。
实际上,当且仅当一个范畴是消瘦范畴 (gaunt category) 时,它才是一个严格范畴 (\cref{ct:gaunt})。
\index{gaunt category}%
\index{category!gaunt}%
大多数情况下,范畴论研究的是范畴,而非严格范畴,但有时人们会考虑严格范畴。
其主要优点在于,严格范畴比等价关系有更严格的“相同”概念,即同构(或者等价地,通过 \cref{ct:cat-eq-iso},即等同关系)。

以下是严格范畴的一个来源。

\begin{eg}\label{ct:mono-cat}
设 $A$ 是一个预范畴,$x:A$ 是一个对象。
则存在一个预范畴 $\mathsf{mono}(A,x)$,如下定义:
\index{monomorphism}
\indexsee{mono}{monomorphism}
\indexsee{monic}{monomorphism}
\begin{itemize}
  \item 其对象由一个对象 $y:A$ 和一个从 $y$ 到 $x$ 的单态射 $m:\hom_A(y,x)$ 组成。
  (通常,$m:\hom_A(y,x)$ 是一个\define{单态射 (monomorphism)}(或称为\define{单态 (monic)})当且仅当 $(m\circ f = m\circ g) \Rightarrow (f=g)$。)
  \item 从 $(y,m)$ 到 $(z,n)$ 的态射是 $A$ 中从 $y$ 到 $z$ 的任意态射(不必尊重 $m$ 和 $n$)。
\end{itemize}
在 $\mathsf{mono}(A,x)$ 中对象的等同 $(y,m)=(z,n)$ 包含对象的等同 $p:y=z$ 和等同 $\trans{p}{m}=n$,根据 \cref{ct:idtoiso-trans},这等价于等同 $m=n\circ \idtoiso(p)$。
由于态射集是集合,这种等同性的类型只是一个单纯命题。
但是由于 $m$ 和 $n$ 是单态射,使得 $m = n\circ f$ 的态射 $f$ 的类型也是一个单纯命题。
因此,如果 $A$ 是一个范畴,那么 $(y,m)=(z,n)$ 只是一个单纯命题,因此 $\mathsf{mono}(A,x)$ 是一个严格范畴。
\end{eg}

这个例子可以对偶化,并以各种方式推广。
以下是严格范畴的一个有趣应用。

\begin{eg}\label{ct:galois}
设 $E/F$ 是有限伽罗瓦扩展 (Galois extension)
\index{Galois!extension}%
的域,$G$ 是其伽罗瓦群 (Galois group)。
\index{Galois!group}%
则存在一个严格范畴,其对象是中间域 $F\subseteq K\subseteq E$,其态射是固定 $F$ 不动的域同态\index{homomorphism!field}(但不必与 $E$ 中的包含映射可交换)。
另一个严格范畴的对象是子群 $H\subseteq G$,其态射是 $G$ 集合 $G/H \to G/K$ 的态射。
伽罗瓦理论的基本定理
\index{fundamental!theorem of Galois theory}%
表明这两个预范畴是同构的(不仅仅是等价的)。
\end{eg}

\index{bargaining|)}%

\section{\texorpdfstring{$\dagger$}{†}-范畴 ($\dagger$-categories)}
\label{sec:dagger-categories}

还值得一提的是一种有用的预范畴,其对象类型不是集合,但它也不是一个范畴。

\begin{defn}\label{ct:dagger-precategory}
一个\define{$\dagger$-预范畴 ($\dagger$-precategory)}
\indexdef{.dagger-precategory@$\dagger$-precategory}%
\indexdef{precategory!.dagger-@$\dagger$-}%
是一个预范畴 $A$,同时具备以下结构:
\begin{enumerate}
  \item 对于每个 $x,y:A$,存在一个函数 $\dgr{(-)}:\hom_A(x,y) \to \hom_A(y,x)$。
  \item 对于所有 $x:A$,我们有 $\dgr{(1_x)} = 1_x$。
  \item 对于所有 $f,g$,我们有 $\dgr{(g\circ f)} = \dgr f \circ \dgr g$。
  \item 对于所有 $f$,我们有 $\dgr{(\dgr f)} = f$。
\end{enumerate}
\end{defn}

\begin{defn}\label{ct:unitary}
在 $\dagger$-预范畴中的态射 $f:\hom_A(x,y)$ 是\define{酉态射 (unitary)}
\indexdef{.dagger-precategory@$\dagger$-precategory!unitary morphism in}%
\indexdef{unitary morphism}%
\indexdef{morphism!unitary}%
\indexdef{isomorphism!unitary}%
如果 $\dgr f \circ f = 1_x$ 并且 $f\circ \dgr f = 1_y$。
\end{defn}

当然,每个酉态射都是同构的,并且酉态射的性质是一个单纯命题。
因此,对于每个 $x,y:A$,我们有从 $x$ 到 $y$ 的酉同构的集合,我们记为 $(x\unitaryiso y)$。

\begin{lem}\label{ct:idtounitary}
如果 $p:(x=y)$,则 $\idtoiso(p)$ 是酉态射。
\end{lem}
\begin{proof}
  通过归纳法,我们可以假设 $p$ 是 $\refl x$。
  但此时 $\dgr{(1_x)} \circ 1_x = 1_x\circ 1_x = 1_x$,类似地。
\end{proof}

\begin{defn}\label{ct:dagger-category}
一个\define{$\dagger$-范畴 ($\dagger$-category)}
\indexdef{.dagger-category@$\dagger$-category}%
是一个 $\dagger$-预范畴,使得对于所有 $x,y:A$,函数
\[ (x=y) \to (x \unitaryiso y) \]
从 \cref{ct:idtounitary} 是一个等价。
\end{defn}

\begin{eg}\label{ct:rel-dagger-cat}
从 \cref{ct:rel} 中的关系范畴 (\urel) 通过定义 $(\dgr R)(y,x) \defeq R(x,y)$ 变成一个 $\dagger$-预范畴。
关系范畴 (\urel) 是一个范畴的证明实际上表明每个同构都是酉的;因此 \urel 也是一个 $\dagger$-范畴。
\end{eg}

\begin{eg}\label{ct:groupoid-dagger-cat}
任何群胚 (groupoid) 通过定义 $\dgr f \defeq \inv{f}$ 变成一个 $\dagger$-范畴。
\end{eg}

\begin{eg}\label{ct:hilb}
令 \uhilb 是以下预范畴:
\begin{itemize}
  \item 其对象是有限维度的\index{finite!-dimensional vector space}带内积 $\langle \blank,\blank\rangle$ 的向量空间\index{vector!space}。
  \item 其态射是任意的线性映射。
  \index{function!linear}%
  \indexsee{linear map}{function, linear}%
\end{itemize}
根据标准的线性代数,任意有限维度内积空间之间的线性映射 $f:V\to W$ 都有一个唯一确定的伴随映射\index{adjoint!linear map} $\dgr f:W\to V$,其特征是 $\langle f v,w\rangle = \langle v,\dgr f w\rangle$。
通过这种方式,\uhilb 变成了一个 $\dagger$-预范畴。
此外,当且仅当线性同构是\define{等距同构 (isometry)}时,它是酉态射,
\indexdef{isometry}%
即 $\langle fv,fw\rangle = \langle v,w\rangle$。
由此可以推断 \uhilb 是一个 $\dagger$-范畴,尽管它不是一个范畴(不是每个线性同构都是酉态射)。
\end{eg}

关于 $\dagger$-范畴的经典理论已经有了相当多的发展\index{mathematics!classical}\index{classical!category theory}。
人们很早就观察到,对于 $\dagger$-范畴的对象,酉同构,而不是任意的同构,是“相同”的正确概念,这在范畴论学者中引起了一些困惑。
同伦类型论通过将 $\dagger$-范畴与严格范畴一样视为一种不同的预范畴,解决了这一问题。

\section{结构同一性原理 (The structure identity principle)}
\label{sec:sip}
\index{structure!identity principle|(}

\emph{结构同一性原理 (structure identity principle)} 是一个非正式的原理,表示同构结构是相同的。我们旨在证明一个可以应用于广泛结构概念的一般抽象结果,其中结构可以是多类的,甚至是依赖类的,无穷的,甚至是高阶的。

最简单的单类结构由没有附加结构的类型组成。单值性公理以一种强烈的形式表达了这种结构概念的结构同一性原理:对于类型 $A,B$,从 $(A=B)\to (\eqv A B)$ 的典型函数是一个等价。

我们从一个预范畴 $X$ 开始。在我们应用于单类一阶结构时,$X$ 将是 $\bbU$-小集的范畴 (\uset%),其中 $\bbU$ 是一个单值类型宇宙。

\begin{defn}\label{ct:sig}
一个\define{结构概念 (notion of structure)}
\indexdef{structure!notion of}%
$(P,H)$ 在 $X$ 上包括以下内容。
\begin{enumerate}
  \item 一个类型族 $P:X_0 \to \type$。
  对于每个 $x:X_0$,$Px$ 的元素称为\define{$(P,H)$-结构 (structures on $x$)}。
  \indexsee{PH-structure@$(P,H)$-structure}{structure}%
  \indexdef{structure!PH@$(P,H)$-}%
  \item 对于 $x,y:X_0$,$f:\hom_X(x,y)$ 和 $\alpha:Px$,$\;\beta:Py$,一个单纯命题
  \[ H_{\alpha\beta}(f)。\]
  如果 $H_{\alpha\beta}(f)$ 为真,我们称 $f$ 是从 $\alpha$ 到 $\beta$ 的\define{$(P,H)$-同态 (homomorphism)}。
  \indexdef{homomorphism!of structures}%
  \indexdef{structure!homomorphism of}%
  \item 对于 $x:X_0$ 和 $\alpha:Px$,我们有 $H_{\alpha\alpha}(1_x)$。\label{item:sigid}
  \item 对于 $x,y,z:X_0$ 和 $\alpha:Px$,$\;\beta:Py$,$\;\gamma:Pz$,如果 $f:\hom_X(x,y)$ 和 $g:\hom_X(y,z)$,我们有\label{item:sigcmp}
  \[ H_{\alpha\beta}(f)\to H_{\beta\gamma}(g)\to H_{\alpha\gamma}(g\circ   f)。\]
\end{enumerate}
当 $(P,H)$ 是一个结构概念时,对于 $\alpha,\beta:Px$ 我们定义
\[ (\alpha\leq_x\beta) \defeq H_{\alpha\beta}(1_x)。\]
根据~\ref{item:sigid} 和~\ref{item:sigcmp},这是一个预序 (\cref{ct:orders}),其对象类型为 $Px$。
如果 $(P,H)$ 是一个\define{标准的结构概念 (standard notion of structure)},则对于 $X$ 上的所有 $x$,这个预序实际上是一个偏序。
\end{defn}

请注意,对于标准的结构概念,每个类型 $Px$ 实际上必须是一个集合。
我们现在定义,对于任何 $(P,H)$ 的结构概念,\define{$(P,H)$-结构的预范畴},
\indexdef{precategory!of PH-structures@of $(P,H)$-structures}%
\indexdef{structure!precategory of PH@precategory of $(P,H)$-}%
$A = \mathsf{Str}_{(P,H)}(X)$。
\begin{itemize}
  \item $A$ 的对象类型是类型 $A_0 \defeq \sm{x:X_0} Px$。
  如果 $a\jdeq (x,\alpha):A_0$,我们可以写作 $|a| \defeq x$。
  \item 对于 $(x,\alpha):A_0$ 和 $(y,\beta):A_0$,我们定义
  \[\hom_A((x,\alpha),(y,\beta)) \defeq \setof{ f:x \to y | H_{\alpha\beta}(f)}。\]
\end{itemize}
合成和身份继承自 $X$;条件~\ref{item:sigid} 和 \ref{item:sigcmp} 确保这些提升到 $A$。

\begin{thm}[结构同一性原理 (Structure identity principle)]\label{thm:sip}
\indexdef{structure!identity principle}%
如果 $X$ 是一个范畴,并且 $(P,H)$ 是 $X$ 上的一个标准结构概念,则预范畴 $\mathsf{Str}_{(P,H)}(X)$ 是一个范畴。
\end{thm}
\begin{proof}
  根据依赖对类型的等同性定义,给出等同性 $(x,\alpha)=(y,\beta)$ 包括以下内容:
  \begin{itemize}
    \item 一个等同性 $p:x=y$,以及
    \item 一个等同性 $\trans{p}{\alpha}=\beta$。
  \end{itemize}
  由于 $P$ 是集合值的,后者只是一个单纯命题。
  另一方面,很容易看出 $\mathsf{Str}_{(P,H)}(X)$ 中的一个同构 $(x,\alpha)\cong (y,\beta)$ 包括以下内容:
  \begin{itemize}
    \item $X$ 中 $x\cong y$ 的一个同构 $f$,使得
    \item $H_{\alpha\beta}(f)$ 和 $H_{\beta\alpha}(\inv f)$ 成立。
  \end{itemize}
  当然,第二项也是一个单纯命题。
  并且由于 $X$ 是一个范畴,函数 $(x=y) \to (x\cong y)$ 是一个等价。
  因此,只需证明对于任意 $p:x=y$ 和任意 $(\alpha:Px)$, $(\beta:Py)$,我们有 $\trans{p}{\alpha}=\beta$ 当且仅当 $H_{\alpha\beta}(\idtoiso (p))$ 和 $H_{\beta\alpha}(\inv{\idtoiso(p)})$ 同时成立。

  “仅当”方向仅是 $\mathsf{Str}_{(P,H)}(X)$ 范畴的函数 $\idtoiso$ 的存在性。
  对于“如果”方向,通过对 $p$ 进行归纳,我们可以假设 $y\jdeq x$ 且 $p\jdeq\refl x$。
  然而,在这种情况下 $\idtoiso (p)\jdeq 1_x$ 因此 $\inv{\idtoiso(p)}=1_x$。
  因此,$\alpha\leq_x \beta$ 和 $\beta\leq_x \alpha$,这意味着 $\alpha=\beta$,因为 $(P,H)$ 是一个标准的结构概念。
\end{proof}

作为一个例子,这种方法提供了一种表达 \cref{ct:functor-cat} 证明的替代方法。

\begin{eg}\label{ct:sip-functor-cat}
设 $A$ 是一个预范畴,$B$ 是一个范畴。
存在一个预范畴 $B^{A_0}$,其对象是从 $A_0$ 到 $B_0$ 的函数,其态射集是从 $F_0:A_0 \to B_0$ 到 $G_0:A_0 \to B_0$ 的 $\gamma:\hom_{B^{A_0}}(F_0, G_0)$ 是一个同构,当且仅当每个分量 $\gamma_a$ 是同构,因此我们有 $\eqv{(F_0 \cong G_0)}{\prd{a:A_0} (F_0 a \cong G_0 a)}$。
此外,$B^{A_0}$ 的 $\idtoiso : (F_0 = G_0) \to (F_0 \cong G_0)$ 映射等同于以下复合函数
\[ (F_0 = G_0) \longrightarrow \prd{a:A_0} (F_0 a  = G_0 a) \longrightarrow \prd{a:A_0} (F_0 a \cong G_0 a) \longrightarrow (F_0 \cong G_0) \]
其中第一个映射是函数外延性给出的等价,第二个因为它是等价的依赖积(因为 $B$ 是一个范畴),第三个如上所述。
因此,$B^{A_0}$ 是一个范畴。

现在我们在 $B^{A_0}$ 上定义一个结构概念,其中 $P(F_0)$ 是扩展 $F_0$ 为函子的操作的类型 $F:\prd{a,a':A_0} \hom_A(a,a') \to \hom_B(F_0 a,F_0 a')$(即保持合成和身份)。
这是一个集合,因为每个 $\hom_B(\blank,\blank)$ 是这样。
给定这样的 $F$ 和 $G$,如果它形成了一个自然变换,我们定义 $\gamma:\hom_{B^{A_0}}(F_0, G_0)$ 为同态。\index{natural!transformation}
在 \cref{ct:functor-precat} 中,我们基本上验证了这是一种结构概念。
此外,如果 $F$ 和 $F'$ 都是 $F_0$ 上的结构,并且恒等是从 $F$ 到 $F'$ 的自然变换,那么对于任何 $f:\hom_A(a,a')$,我们有 $F'f = F'f \circ 1_{F_0 a} = 1_{F_0 a}\circ F f = F f$。
应用函数外延性,我们得出结论 $F = F'$。
因此,我们有一个\emph{标准}结构概念,因此根据 \cref{thm:sip},预范畴 $B^A$ 是一个范畴。
\end{eg}

作为另一个例子,我们考虑一阶签名的结构范畴。
我们定义一个\define{一阶签名 (first-order signature)},
\indexdef{first-order!signature}%
\indexdef{signature!first-order}%
$\Omega$,其由函数符号 $\Omega_0$ 和关系符号 $\Omega_1$ 组成,$\omega:\Omega_0$ 和 $\omega:\Omega_1$,每个符号都有一个为集合的元数\index{arity} $|\omega|$。
一个\define{$\Omega$-结构 ($\Omega$-structure)}%
\indexdef{structure!Omega@$\Omega$-}%
\indexsee{omega-structure@$\Omega$-structure}{structure}%
$a$ 由一个集合 $|a|$ 以及为每个函数符号 $\omega$ 分配的一个 $|\omega|$ 元函数 $\omega^a:|a|^{|\omega|}\to |a|$ 和为每个关系符号 $\omega$ 分配的一个 $|\omega|$ 元关系 $\omega^a$ 组成,赋予每个 $x:|a|^{|\omega|}$ 一个单纯命题 $\omega^ax$。
并且对于给定的 $\Omega$-结构 $a,b$,如果它保持结构,即对于每个签名符号 $\omega$ 和每个 $x:|a|^{|\omega|}$,
\begin{enumerate}
  \item 如果 $\omega:\Omega_0$,则 $f(\omega^ax) = \omega^b(f\circ x)$,并且
  \item 如果 $\omega:\Omega_1$,则 $\omega^ax\to\omega^b(f\circ x)$。
\end{enumerate}
请注意,每个 $x:|a|^{|\omega|}$ 是一个函数 $x:|\omega|\to |a|$ 使得 $f\circ x : b^\omega$。

现在我们假设给定一个(单值的)宇宙 $\bbU$ 和一个 $\bbU$-小的签名 $\Omega$;即 $|\Omega|$ 是一个 $\bbU$-小集合,并且对于每个 $\omega:|\Omega|$,集合 $|\omega|$ 是 $\bbU$-小集合。
然后我们有 $\bbU$-小集合的范畴 $\uset_\bbU$。我们希望定义 $\uset_\bbU$ 上的 $\bbU$-小 $\Omega$-结构的预范畴,并使用 \cref{thm:sip} 来证明它是一个范畴。

我们使用一阶签名 $\Omega$ 给我们一个 $\uset_\bbU$ 上的标准结构概念 $(P,H)$。

\begin{defn}\label{defn:fo-notion-of-structure}
\mbox{}
\begin{enumerate}
  \item 对于每个 $\bbU$-小集合 $x$ 定义
  \[ Px \defeq P_0x\times P_1x。\]
  这里
  %
  \begin{align*}
    P_0x &\defeq \prd{\omega:\Omega_0} x^{|\omega|}\to x, \mbox{ 和 } \\
    P_1x &\defeq \prd{\omega:\Omega_1} x^{|\omega|}\to \propU,
  \end{align*}
  \item 对于 $\bbU$-小集合 $x,y$ 和
  $\alpha:P^\omega x,\;\beta:P^\omega y,\; f:x\to y$,定义
  \[ H_{\alpha\beta}(f) \defeq H_{0,\alpha\beta}(f)\wedge H_{1,\alpha\beta}(f)。\]
  这里
  \begin{align*}
    H_{0,\alpha\beta}(f) &\defeq
    \fall{\omega:\Omega_0}{u:x^{|\omega|}} f(\alpha u)=\;\beta(f\circ u),
    \mbox{ 和 }\\
    H_{1,\alpha\beta}(f) &\defeq
    \fall{\omega:\Omega_1}{u:x^{|\omega|}} \alpha u\to\beta(f\circ u)。
  \end{align*}
\end{enumerate}
\end{defn}

现在例行公事地检查 $(P,H)$ 是 $\uset_\bbU$ 上的标准结构概念,因此我们可以使用 \cref{thm:sip} 得出 $\bbU$-小 $\Omega$-结构的预范畴 $Str_{(P,H)}(\uset_\bbU)$ 是一个范畴。只需观察到这本质上与 $\uset_\bbU$ 上的 $\bbU$-小 $\Omega$-结构的预范畴相同。
\index{structure!identity principle|)}
\section{Rezk 完备 (The Rezk completion)}
\label{sec:rezk}

在本节中,我们将给出一种将预范畴替换为范畴的通用方法。实际上,我们将给出两种方法。这两种方法都依赖于这样一个事实:“范畴将弱等价视为等价”。

为了证明这一点,我们首先引入一些完全标准的范畴论引理,这些引理经过仔细表述,以确保我们正确使用了 $\truncf{-1}$ 的消元器。如果我们想要避免选择公理,在经典\index{mathematics!classical}\index{classical!category theory}范畴论中也需要同样谨慎地处理:任何时候我们想要定义一个函数,我们都需要以某种方式唯一地描述其值。

\begin{lem}\label{ct:esosurj-postcomp-faithful}
如果 $A,B,C$ 是预范畴,并且 $H:A\to B$ 是一个本质上满的函子,那么 $(\blank\circ H):C^B \to C^A$ 是忠实的。
\end{lem}
\begin{proof}
  设 $F,G:B\to C$,并且 $\gamma,\delta:F\to G$ 满足 $\gamma H = \delta H$;我们必须证明 $\gamma=\delta$。
  因此设 $b:B$;我们想要证明 $\gamma_b=\delta_b$。
  这是一个单纯命题,因此由于 $H$ 本质上满,我们可以假设给定 $a:A$ 和同构 $f:Ha\cong b$。
  但是现在我们有
  \[ \gamma_b = G(f) \circ \gamma_{Ha} \circ F(\inv{f})
  = G(f) \circ \delta_{Ha} \circ F(\inv{f})
  = \delta_b。\qedhere
  \]
\end{proof}

\begin{lem}\label{ct:esofull-precomp-ff}
如果 $A,B,C$ 是预范畴,并且 $H:A\to B$ 是本质上满且充满的,那么 $(\blank\circ H):C^B \to C^A$ 是充分忠实的。
\end{lem}
\begin{proof}
  现在还需要证明充满性。
  因此,设 $F,G:B\to C$ 和 $\gamma:FH \to GH$。
  我们声称对于任意 $b:B$,类型
  \begin{equation}\label{eq:fullprop}
  \sm{g:\hom_C(Fb,Gb)} \prd{a:A}{f:Ha\cong b} (\gamma_a =  \inv{Gf}\circ g\circ Ff)
  \end{equation}
  是可约的。
  由于可约性是一个单纯命题,并且 $H$ 是本质上满的,我们可以假设给定 $a_0:A$ 和 $h:Ha_0\cong b$。

  现在令 $g\defeq Gh \circ \gamma_{a_0} \circ \inv{Fh}$。
  那么给定任意其他的 $a:A$ 和 $f:Ha\cong b$,我们必须证明 $\gamma_a =  \inv{Gf}\circ g\circ Ff$。
  由于 $H$ 是充满的,存在一个态射 $k:\hom_A(a,a_0)$ 使得 $Hk = \inv{h}\circ f$。
  并且由于我们的目标是一个单纯命题,我们可以假设给定某个这样的 $k$。
  然后我们有
  \begin{align*}
    \gamma_a &= \inv{GHk}\circ \gamma_{a_0} \circ FHk\\
    &= \inv{Gf} \circ Gh \circ \gamma_{a_0} \circ \inv{Fh} \circ Ff\\
    &= \inv{Gf}\circ g\circ Ff.
  \end{align*}
  因此,~\eqref{eq:fullprop} 是可居住的。
  现在还需要证明它是一个单纯命题。
  设 $g,g':\hom_C(Fb, Gb)$,并且对于所有 $a:A$ 和 $f:Ha\cong b$,我们有 $(\gamma_a =  \inv{Gf}\circ g\circ Ff)$ 和 $(\gamma_a =  \inv{Gf}\circ g'\circ Ff)$。
  依赖积类型是单纯命题,因此我们只需要证明 $g=g'$。
  但这是一个单纯命题,因此我们可以假设 $a_0:A$ 和 $h:Ha_0\cong b$,在这种情况下我们有
  \[ g = Gh \circ \gamma_{a_0} \circ \inv{Fh} = g'。\]

  这证明了对于所有 $b:B$,~\eqref{eq:fullprop} 是可约的。
  现在我们通过为每个 $b$ 取~\eqref{eq:fullprop} 中唯一的 $g$ 来定义 $\delta:F\to G$。
  为了证明这是自然的,假设给定 $f:\hom_B(b,b')$;我们必须证明 $Gf \circ \delta_b = \delta_{b'}\circ Ff$。
  和以前一样,我们可以假设 $a:A$ 和 $h:Ha\cong b$,以及 $a':A$ 和 $h':Ha'\cong b'$。
  由于 $H$ 既是充满的也是本质上满的,我们还可以假设 $k:\hom_A(a,a')$ 并且 $Hk = \inv{h'}\circ f\circ h$。

  由于 $\gamma$ 是自然的,$GHk\circ \gamma_a = \gamma_{a'} \circ FHk$。
  使用 $\delta$ 的定义,我们有
  \begin{align*}
    Gf \circ \delta_b
    &= Gf \circ Gh \circ \gamma_a \circ \inv{Fh}\\
    &= Gh' \circ GHk\circ \gamma_a \circ \inv{Fh}\\
    &= Gh' \circ \gamma_{a'} \circ FHk \circ \inv{Fh}\\
    &= Gh' \circ \gamma_{a'} \circ \inv{Fh'} \circ Ff\\
    &= \delta_{b'} \circ Ff.
  \end{align*}
  因此,$\delta$ 是自然的。
  最后,对于任意 $a:A$,将 $\delta_{Ha}$ 的定义应用于 $a$ 和 $1_a$,我们得到 $\gamma_a = \delta_{Ha}$。
  因此,$\delta \circ H = \gamma$。
\end{proof}

定理的其余部分几乎完全按照相同的思路进行,唯一不同的是在一个关键步骤中插入了范畴的性质,我们在下面用斜体强调了这一点。这是在定义一个到\emph{对象}的函数时不使用选择公理的步骤,因此我们必须小心定义对象“唯一确定”的含义。在经典\index{mathematics!classical}\index{classical!category theory}范畴论中,我们只能说这个对象在唯一同构意义上是确定的,但在集合论基础中,这样的唯一性并不足以在不引入选择公理的情况下定义一个函数。然而,在单值基础中,如果 $C$ 是一个范畴,那么同构就是等同,并且我们有适当的唯一性(即存在于一个可约空间中)。

\index{weak equivalence!of precategories|(}%

\begin{thm}\label{ct:cat-weq-eq}
如果 $A,B$ 是预范畴,$C$ 是一个范畴,并且 $H:A\to B$ 是一个弱等价,那么 $(\blank\circ H):C^B \to C^A$ 是一个同构。
\end{thm}
\begin{proof}
  由 \cref{ct:functor-cat},$C^B$ 和 $C^A$ 是范畴。
  因此,根据 \cref{ct:eqv-levelwise},我们只需证明 $(\blank\circ H)$ 是一个等价。
  但是,由于我们从前面两个引理中知道它是充分忠实的,根据 \cref{ct:catweq},我们只需证明它是本质上满的。
  因此,假设 $F:A\to C$;我们希望仅存在一个 $G:B\to C$ 使得 $GH\cong F$。

  对于每个 $b:B$,令 $X_b$ 为一个类型,其元素由以下组成:
  \begin{enumerate}
    \item 一个元素 $c:C$;以及
    \item 对于每个 $a:A$ 和 $h:Ha\cong b$,一个同构 $k_{a,h}:Fa\cong c$;满足\label{item:eqvprop2}
    \item 对于~\ref{item:eqvprop2} 中的每个 $(a,h)$ 和 $(a',h')$ 以及每个满足 $h'\circ Hf = h$ 的态射 $f:\hom_A(a,a')$,我们有 $k_{a',h'}\circ Ff = k_{a,h}$。\label{item:eqvprop3}
  \end{enumerate}
  我们声称对于任意 $b:B$,类型 $X_b$ 是可约的。
  由于这是一个单纯命题,我们可以假设给定 $a_0:A$ 和 $h_0:Ha_0 \cong b$。
  令 $c^0\defeq Fa_0$。
  接下来,给定 $a:A$ 和 $h:Ha\cong b$,由于 $H$ 是充分忠实的,存在一个唯一的同构 $g_{a,h}:a\to a_0$,使得 $Hg_{a,h} = \inv{h_0}\circ h$;定义 $k^0_{a,h} \defeq Fg_{a,h}$。
  最后,如果 $h'\circ Hf = h$,那么 $\inv{h_0}\circ h'\circ Hf = \inv{h_0}\circ h$,因此 $g_{a',h'} \circ f = g_{a,h}$,从而 $k^0_{a',h'}\circ Ff = k^0_{a,h}$。
  因此,$X_b$ 是可居住的。

  现在假设给定另一个 $(c^1,k^1): X_b$。
  那么 $k^1_{a_0,h_0}:c^0 \jdeq Fa_0 \cong c^1$。
  \emph{由于 $C$ 是一个范畴,我们有 $p:c^0=c^1$ 并且 $\idtoiso(p) = k^1_{a_0,h_0}$。}
  并且对于任意 $a:A$ 和 $h:Ha\cong b$,通过~\ref{item:eqvprop3},对于 $(c^1,k^1)$ 和 $f\defeq g_{a,h}$,我们有
  \[k^1_{a,h} = k^1_{a_0,h_0} \circ k^0_{a,h} = \trans{p}{k^0_{a,h}}\]
  这提供了等同性 $(c^0,k^0)=(c^1,k^1)$ 所需的数据,从而完成了 $X_b$ 可约性的证明。

  现在,由于对于每个 $b$,$X_b$ 是可约的,类型 $\prd{b:B} X_b$ 也是可约的。
  特别是,它是可居住的,因此我们有一个函数,将每个 $b:B$ 映射到一个 $c$ 和一个 $k$。
  定义 $G_0(b)$ 为这个 $c$;这给出了一个函数 $G_0 :B_0 \to C_0$。

  接下来我们需要定义 $G$ 在态射上的作用。
  对于每个 $b,b':B$ 和 $f:\hom_B(b,b')$,令 $Y_f$ 为一个类型,其元素由以下组成:
  \begin{enumerate}[resume]
    \item 一个态射 $g:\hom_C(Gb,Gb')$,使得
    \item 对于每个 $a:A$ 和 $h:Ha\cong b$,以及每个 $a':A$ 和 $h':Ha'\cong b'$,以及任意 $\ell:\hom_A(a,a')$,我们有\label{item:eqvprop5}
    \[ (h' \circ H\ell = f \circ h)
    \to
    (k_{a',h'} \circ F\ell = g\circ k_{a,h})。 \]
  \end{enumerate}
  我们声称对于任意 $b,b'$ 和 $f$,类型 $Y_f$ 是可约的。
  由于这是一个单纯命题,我们可以假设给定 $a_0:A$ 和 $h_0:Ha_0\cong b$,以及每个 $a'_0:A$ 和 $h'_0:Ha'_0\cong b'$。
  然后由于 $H$ 是充分忠实的,存在一个唯一的 $\ell_0:\hom_A(a_0,a_0')$ 使得 $h'_0 \circ H\ell_0 = f \circ h_0$。
  定义 $g_0 \defeq k_{a_0',h_0'} \circ F \ell_0 \circ \inv{(k_{a_0,h_0})}$。

  现在对于任意 $a,h,a',h'$,和 $\ell$ 使得 $(h' \circ H\ell = f \circ h)$,我们有 $\inv{h}\circ h_0:Ha_0\cong Ha$,因此存在一个唯一的 $m:a_0\cong a$ 使得 $Hm = \inv{h}\circ h_0$,因此 $h\circ Hm = h_0$。
  类似地,我们有一个唯一的 $m':a_0'\cong a'$ 使得 $h'\circ Hm' = h_0'$。
  现在通过~\ref{item:eqvprop3},我们有 $k_{a,h}\circ Fm = k_{a_0,h_0}$ 和 $k_{a',h'}\circ Fm' = k_{a_0',h_0'}$。
  我们还有
  \begin{align*}
    Hm' \circ H\ell_0
    &= \inv{(h')} \circ h_0' \circ H\ell_0\\
    &= \inv{(h')} \circ f \circ h_0\\
    &= \inv{(h')} \circ f \circ h \circ \inv{h} \circ h_0\\
    &= H\ell \circ Hm
  \end{align*}
  因此,由于 $H$ 是充分忠实的,$m'\circ \ell_0 = \ell\circ m$。
  最后,我们可以计算出
  \begin{align*}
    g_0 \circ k_{a,h}
    &= k_{a_0',h_0'} \circ F \ell_0 \circ \inv{(k_{a_0,h_0})} \circ k_{a,h}\\
    &= k_{a_0',h_0'} \circ F \ell_0 \circ \inv{Fm}\\
    &= k_{a_0',h_0'} \circ \inv{(Fm')} \circ F\ell\\
    &= k_{a',h'}\circ F\ell。
  \end{align*}
  这完成了 $Y_f$ 可居住性的证明。
  为了证明它是可约的,由于态射集是集合,因此只需取另一个 $g_1:\hom_C(Gb,Gb')$ 满足~\ref{item:eqvprop5} 并证明 $g_0=g_1$。
  然而,我们仍然保留了指定的 $a_0,h_0,a_0',h_0',\ell_0$,并且~\ref{item:eqvprop5} 表明 $g_0$ 和 $g_1$ 必须都等于 $k_{a_0',h_0'} \circ F \ell_0 \circ \inv{(k_{a_0,h_0})}$。

  这完成了对于每个 $b,b':B$ 和 $f:\hom_B(b,b')$,$Y_f$ 是可约的证明。
  因此,有一个函数将每个 $f$ 映射到其唯一的居住者;将此函数记为 $G_{b,b'}:\hom_B(b,b') \to \hom_C(Gb,Gb')$。
  $G$ 是一个函子的证明是直截了当的;在每种情况下,我们都可以选择 $a,h$ 并应用~\ref{item:eqvprop5}。

  最后,对于任意 $a_0:A$,定义 $c\defeq Fa_0$ 并且 $k_{a,h}\defeq F g$,其中 $g:\hom_A(a,a_0)$ 是与 $Hg = h$ 唯一同构的同构,给出了 $X_{Ha_0}$ 的一个元素。
  因此,它等于指定的一个;因此 $GHa=Fa$。
  类似地,对于 $f:\hom_A(a_0,a_0')$,我们可以通过沿着这些等同性传递来定义 $Y_{Hf}$ 的一个元素,因此它必须等于指定的一个。
  因此,我们有 $GH=F$,从而 $GH\cong F$ 如预期。
\end{proof}

\index{universal!property!of Rezk completion}%
因此,如果一个预范畴 $A$ 允许一个弱等价函子 $A\to \widehat{A}$ 映射到一个范畴,那么它就是范畴的“反射”:从 $A$ 到范畴的任何函子都将通过 $\widehat{A}$ 本质上唯一地分解。
我们现在给出两个此类弱等价的构造。

\indexsee{Rezk completion}{completion, Rezk}%
\index{completion!Rezk|(defstyle}%

\begin{thm}\label{thm:rezk-completion}
对于任何预范畴 $A$,存在一个范畴 $\widehat A$ 和一个弱等价 $A\to\widehat{A}$。
\end{thm}

\begin{proof}[第一个证明]
  令 $\widehat{A}_0 \defeq \setof{ F:\uset^{A\op} | \exis{a:A} (\y a \cong F)}$,其态射集继承自 $\uset^{A\op}$。
  然后,包含映射 $\widehat{A} \to \uset^{A\op}$ 是充分忠实的,并且在对象上是嵌入的。
  由于 $\uset^{A\op}$ 是一个范畴(由 \cref{ct:functor-cat},因为 \uset 是由单值性公理给出的),$\widehat A$ 也是一个范畴。

  令 $A\to\widehat A$ 为 Yoneda 嵌入。
  由 \cref{ct:yoneda-embedding},这条映射是充分忠实的,并且根据 $\widehat{A}_0$ 的定义本质上满。
  因此它是一个弱等价。
\end{proof}

这个证明非常简洁,但它的缺点是它增加了宇宙级别。
如果 $A$ 是 $\bbU$ 宇宙中的一个范畴,那么在这个证明中 \uset 必须至少与 $\uset_\bbU$ 一样大。
然后 $\uset_\bbU$ 和 $(\uset_\bbU)^{A\op}$ 本身不是 $\bbU$ 中的范畴,而只是在更高的宇宙中是范畴,并且\emph{先验地} $\widehat A$ 也是如此。
人们可以想象一个可以处理这个问题的重置公理,但也可以使用高阶归纳类型直接构造。

\begin{proof}[第二个证明]
  我们定义一个高阶归纳类型 $\widehat A_0$,其具有以下构造函数:
  \begin{itemize}
    \item 一个函数 $i:A_0 \to \widehat A_0$。
    \item 对于每个 $a,b:A$ 和 $e:a\cong b$,一个等同性 $je:\id{ia}{ib}$。
    \item 对于每个 $a:A$,一个等同性 $\id{j(1_a)}{\refl{ia}}$。
    \item 对于每个 $(a,b,c:A)$,$(f:a\cong b)$,和 $(g:b\cong c)$,一个等同性 $\id{j(g \circ f)}{j(f)\ct j(g)}$。
    \item 1-截断:对于所有 $x,y:\widehat A_0$ 和 $p,q:\id x y$ 以及 $r,s:\id p q$,一个等同性 $\id r s$。
  \end{itemize}
  注意,对于任何 $a,b:A$ 和 $p:\id a b$,我们有 $\id{j(\idtoiso(p))}{\map i p}$。
  这通过对 $p$ 的路径归纳得出,并且通过第三个构造函数给出。

  类型 $\widehat A_0$ 将是 $\widehat A$ 的对象类型;我们现在构建其余的结构。
  (以下证明是那种可以从计算机证明助手的帮助中获益匪浅的:它是广泛而浅显的,涉及许多简短的情况需要考虑,大部分工作是写下需要检查的内容。)

  \mentalpause

  \emph{步骤 1:}我们通过对 $\widehat A_0$ 进行双重归纳定义一个族 $\hom_{\widehat A}:\widehat A_0\to \widehat A_0 \to \set$。
  由于 \set 是一个 1-类型,我们可以忽略 1-截断构造函数。
  当 $x$ 和 $y$ 是 $ia$ 和 $ib$ 时,我们取 $\hom_{\widehat A}(ia,ib) \defeq \hom_A(a,b)$。
  还需要考虑其他可能的构造对。

  让我们首先保持 $x=ia$ 固定。
  如果 $y$ 随着等同性 $je:\id{ib}{ib'}$ 变化,对于某些 $e:b\cong b'$,我们需要一个等同性 $\id{\hom_A(a,b)}{\hom_A(a,b')}$。
  通过单值性,给出一个等价 $\eqv{\hom_A(a,b)}{\hom_A(a,b')}$ 就足够了。
  我们取这个函数 $(e\circ \blank ):\hom_A(a,b)\to \hom_A(a,b')$。
  要证明这是一个等价,我们将其逆定义为 $(\inv e\circ \blank )$,并通过 $e$ 的逆是 $e$ 在 $A$ 中的逆的事实来证明其逆性。

  当 $y$ 随着等同性 $\id{j(1_b)}{\refl{ib}}$ 变化时,我们需要一个等同性 $\id{(1_b\circ \blank )}{\refl{\hom_A(a,b)}}$;这由预范畴的身份公理 $\id{1_b\circ g}{g}$ 推导而来。
  类似地,当 $y$ 随着等同性 $\id{j(g\circ f)}{j(f)\ct j(g)}$ 变化时,我们需要一个等同性 $\id{((g\circ f)\circ \blank )}{(g\circ (f\circ \blank ))}$,这由结合律推导而来。
  % 最后,当 $y$ 随着 1-截断构造变化时,我们只需要观察到 \set 是 1-截断的。

  现在我们考虑 $x$ 的其他构造。
  假设 $x$ 随着等同性 $j(e):\id{ia}{ia'}$ 变化,对于某些 $e:a \cong a'$;我们再次必须处理 $y$ 的所有构造。
  如果 $y$ 是 $ib$,那么我们需要一个等同性 $\id{\hom_A(a,b)}{\hom_A(a',b)}$。
  通过单值性,这可能来自一个等价,并且我们可以使用 $(\blank\circ \inv e)$,逆为 $(\blank\circ e)$。

  仍然是 $x$ 随着 $j(e)$ 变化,假设现在 $y$ 也随 $j(f)$ 变化,对于某些 $f:b\cong b'$。
  那么我们需要知道两个级联的等同性
  \begin{gather*}
    \hom_A(a,b) = \hom_A(a',b) = \hom_A(a',b') \mathrlap{\qquad\text{和}}\\
    \hom_A(a,b) = \hom_A(a,b') = \hom_A(a',b')
  \end{gather*}
  是相同的。
  这由结合律得出:$(f\circ \blank)\circ \inv e = f\circ (\blank\circ \inv e)$。
  $y$ 的其他两个构造是平凡的,因为它们是集合中的 2-重等同性。

  对于 $x$ 的接下来的两个构造,$y$ 的所有构造除了第一个都是平凡的。
  当 $x$ 随着 $j(1_a)=\refl{ia}$ 变化,并且 $y$ 是 $ib$ 时,我们再次使用身份公理。
  类似地,当 $x$ 随着 $\id{j(g\circ f)}{j(f)\ct j(g)}$ 变化时,我们再次使用结合律。
  这完成了 $\hom_{\widehat A}:\widehat A_0 \to \widehat A_0 \to \set$ 的构造。

  \mentalpause

  \emph{步骤 2:}我们通过对 $\widehat A_0$ 进行归纳来给出 $\widehat A$ 上的预范畴结构。
  % 读者可能到此感到厌倦,因此我们省略了细节。
  我们现在正在向集合($\widehat A$ 的态射集)消元,因此除了前两个构造外,所有构造都很容易处理。

  对于恒等式,如果 $x$ 是 $ia$,那么我们有 $\hom_{\widehat A}(x,x) \jdeq \hom_A(a,a)$ 并且我们定义 $1_x \defeq 1_{ia}$。
  如果 $x$ 随着 $je$ 变化,对于 $e:a\cong a'$,我们必须证明 $\transfib{x\mapsto \hom_{\widehat A}(x,x)}{je}{1_{ia}} = 1_{ia'}$。
  但根据 $\hom_{\widehat A}$ 的定义,沿着 $je$ 传递等同于与 $e$ 和 $\inv e$ 复合,并且我们有 $e\circ 1_{ia} \circ \inv{e} = 1_{ia'}$。

  对于复合,如果 $x,y,z$ 分别是 $ia,ib,ic$,那么 $\hom_{\widehat A}$ 简化为 $\hom_A$,我们可以定义 $\widehat A$ 中的复合作为 $A$ 中的复合。
  当 $x$、$y$ 或 $z$ 随着 $je$ 变化时,我们验证以下等同性:
  \begin{align*}
    e \circ (g\circ f) &= (e\circ g) \circ f,\\
    g\circ f &= (g\circ \inv e) \circ (e\circ f),\\
    (g\circ f) \circ \inv e &= g \circ (f\circ \inv e)。
  \end{align*}
  最后,结合性和单性公理是单纯命题,因此除了第一个构造外,所有构造都是平凡的。
  但是在这种情况下,我们有 $A$ 中的相应公理。

  \mentalpause

  \emph{步骤 3:}我们展示 $\widehat A$ 是一个范畴。
  即我们必须证明对于所有 $x,y:\widehat A$,函数 $\idtoiso:(x=y) \to (x\cong y)$ 是一个等价。
  首先我们定义,对于所有 $x,y:\widehat A$,一个函数 $k_{x,y}:(x\cong y) \to (x=y)$,通过归纳。
  如前所述,由于我们的目标是一个集合,除了前两个构造外,所有构造都是平凡的。

  当 $x$ 和 $y$ 分别是 $ia$ 和 $ib$ 时,我们有 $\hom_{\widehat A}(ia,ib)\jdeq \hom_A(a,b)$,其复合和恒等式也继承了,因此 $(ia\cong ib)$ 等价于 $(a\cong b)$。
  但现在我们有构造 $j:(a\cong b) \to (ia=ib)$。

  接下来,如果 $y$ 随着 $j(e)$ 变化,对于某些 $e:b\cong b'$,我们必须证明对于 $f:a\cong b$,我们有 $j(\trans{j(e)}{f}) = j(f) \ct j(e)$。
  但是通过 $\hom_{\widehat A}$ 在等同性上的定义,沿着 $j(e)$ 传递等同于与 $e$ 后复合,因此这个等同性由 $\widehat A_0$ 的最后一个构造得出。
  当 $x$ 随着 $j(e)$ 变化对于 $e:a\cong a'$ 时,其余情况类似。
  这完成了 $k:\prd{x,y:\widehat A_0} (x\cong y) \to (x=y)$ 的定义。

  现在我们必须展示的一件事是,如果 $p:x=y$,那么 $k(\idtoiso(p))=p$。
  通过对 $p$ 的归纳,我们可以假设它是 $\refl x$,因此 $\idtoiso(p)\jdeq 1_x$。
  现在我们通过对 $x:\widehat A_0$ 的归纳进行论证,并且由于我们的目标是一个单纯命题(因为 $\widehat A_0$ 是一个 1-类型),除了第一个构造外,所有构造都是平凡的。
  但是如果 $x$ 是 $ia$,那么 $k(1_{ia}) \jdeq j(1_a)$,它等于 $\refl{ia}$ 由 $\widehat A_0$ 的第三个构造得出。

  为了完成 $\widehat A$ 是一个范畴的证明,我们必须证明如果 $f:x\cong y$,那么 $\idtoiso(k(f))=f$。
  通过归纳我们可以假设 $x$ 和 $y$ 分别是 $ia$ 和 $ib$,在这种情况下 $f$ 必须来自一个同构 $g:a\cong b$ 并且我们有 $k(f)\jdeq j(g)$。
  然而,对于任何 $p$ 我们有 $\idtoiso(p) = \trans{p}{1}$,特别地 $\idtoiso (j(g)) = \trans{j(g)}{1_{ia}}$。
  并且通过 $\hom_{\widehat A}$ 在等同性上的定义,这个传递等同于将 $1_{ia}$ 与等价 $g$ 复合,因此等于 $g$。

  \index{encode-decode method}%
  注意这一步与在 \cref{sec:compute-coprod,sec:compute-nat,cha:homotopy} 中使用的编码-解码方法\index{encode-decode method}的相似性。
  再次,我们通过递归地定义一族代码(这里是 $(x,y)\mapsto (x\cong y)$)和编码与解码函数,来描述一个高阶归纳类型的身份类型(这里是 $\widehat A_0$),并在 $\widehat A_0$ 和路径上进行归纳。

  \mentalpause

  \emph{步骤 4:}我们定义一个弱等价 $I:A \to \widehat A$。
  我们取 $I_0 \defeq i : A_0 \to \widehat A_0$,并且通过 $\hom_{\widehat A}$ 的构造,我们有函数 $I_{a,b}:\hom_A(a,b) \to \hom_{\widehat A}(Ia,Ib)$ 形成函子 $I:A \to \widehat A$。
  这个函子通过构造是充分忠实的,因此剩下的就是证明它本质上是满的。
  即对于所有 $x:\widehat A$ 我们希望仅存在一个 $a:A$ 使得 $Ia\cong x$。
  如前所述,我们通过对 $x$ 的归纳进行论证,并且由于目标是一个单纯命题,除了第一个构造外,所有构造都是平凡的。
  但是如果 $x$ 是 $ia$,那么当然我们有 $a:A$ 并且 $Ia\jdeq ia$,因此 $Ia \cong ia$。
  (请注意,如果我们试图证明 $I$ 是\emph{分裂}本质上满的,我们将陷入困境,因为我们对 $A_0$ 中的等同性一无所知,因此无法处理进一步的构造。)
\end{proof}

我们将构造 $A\mapsto \widehat A$ 称为\define{Rezk 完备 (Rezk completion)},
尽管也有一个(来自高阶拓扑语义的)论点
\index{.infinity1-topos@$(\infty,1)$-topos}%
称其为\define{堆栈完备 (stack completion)}。
\index{stack}%
\index{completion!Rezk|)}%

我们已经看到,大多数在实践中出现的预范畴都是范畴,因为它们是从 \uset 构造的,而 \uset 由于单值性公理而是一个范畴。
然而,在某些情况下,为了获得一个范畴,Rezk 完备是必要的。

\begin{eg}\label{ct:rezk-fundgpd-trunc1}
回想 \cref{ct:fundgpd},对于任意类型 $X$,存在一个预群范畴,其对象类型为 $X$,且 $\hom(x,y) \defeq \pizero{x=y}$。
\indexdef{fundamental!groupoid}%
\index{fundamental!pregroupoid}%
\indexsee{groupoid!fundamental}{fundamental group\-oid}%
其 Rezk 完备是 $X$ 的\emph{基本群范畴 (fundamental groupoid)}。
回想群范畴等价于 1-类型,不难将此群范畴识别为 $\trunc1X$。
\end{eg}

\begin{eg}\label{ct:hocat}
回想 \cref{ct:hoprecat},存在一个预范畴,其对象类型为 \type 且 $\hom(X,Y) \defeq \pizero{X\to Y}$。
其 Rezk 完备可以称为\define{类型的同伦范畴 (homotopy category of types)}。
\index{category!of types}%
\index{homotopy!category of types@(pre)category of types}%
其对象类型可以识别为 $\trunc1\type$(见 \cref{ct:ex:hocat})。
\end{eg}

Rezk 完备还允许我们表明,“范畴”的概念是由“预范畴的弱等价”的概念所决定的。
因此,后者不可避免地决定了前者。

\begin{thm}\label{ct:weq-iso-precat-cat}
一个预范畴 $C$ 是一个范畴,当且仅当对于每个预范畴的弱等价 $H:A\to B$,诱导的函子 $(\blank\circ H):C^B \to C^A$ 是一个预范畴的同构。
\end{thm}
\begin{proof}
  “只有如果” 是 \cref{ct:cat-weq-eq}。
  在另一个方向上,令 $H$ 为 $I:A\to\widehat A$。
  然后由于 $(\blank\circ I)_0$ 是等价的,存在 $R:\widehat A\to A$ 使得 $RI=1_A$。
  因此 $IRI=I$,但再次由于 $(\blank\circ I)_0$ 是等价的,这意味着 $IR =1_{\widehat A}$。
  根据 \cref{ct:isoprecat}\ref{item:ct:ipc3},$I$ 是预范畴的同构。
  但是由于 $\widehat A$ 是一个范畴,因此 $A$ 也是一个范畴。
\end{proof}

\index{weak equivalence!of precategories|)}%


\newpage

\sectionNotes

The original definition of categories, of course, was in set-theoretic foundations, so that the collection of objects of a category formed a set (or, for large categories, a class).
Over time, it became clear that all ``category-theoretic'' properties of objects were invariant under isomorphism, and that equality of objects in a category was not usually a very useful notion.
Numerous authors~\cite{blanc:eqv-log,freyd:invar-eqv,makkai:folds,makkai:comparing} discovered that a dependently typed logic enabled formulating the definition of category without invoking any notion of equality for objects, and that the statements provable in this logic are precisely the ``category-theoretic'' ones that are invariant under isomorphism.
\index{evil}%

Although most of category theory appears to be invariant under isomorphism of objects and under equivalence of categories, there are some interesting exceptions, which have led to philosophical discussions about what it means to be ``category-theoretic''.
For instance, \cref{ct:galois} was brought up by Peter May on the categories mailing list in May 2010, as a case where it matters that two categories (defined as usual in set theory) are isomorphic rather than only equivalent.
The case of $\dagger$-categories was also somewhat confounding to those advocating an isomorphism-invariant version of category theory, since the ``correct'' notion of sameness between objects of a $\dagger$-category is not ordinary isomorphism but \emph{unitary} isomorphism.
\index{isomorphism!invariance under}%

Categories satisfying the ``saturation'' or ``univalence'' principle as in \cref{ct:category} were first considered by Hofmann and Streicher~\cite{hs:gpd-typethy}.
The condition then occurred independently to Voevodsky, Shulman, and perhaps others around the same time several years later, and was formalized by Ahrens and Kapulkin~\cite{aks:rezk}.
This framework puts all the above examples in a unified context: some precategories are categories, others are strict categories, and so on.
A general theorem that ``isomorphism implies equality'' for a large class of algebraic structures (assuming the univalence axiom) was proven by Coquand and Danielsson; the formulation of the structure identity principle in \cref{sec:sip} is due to Aczel.

Independently of philosophical considerations about category theory, Rezk~\cite{rezk01css} discovered that when defining a notion of $(\infty,1)$-cat\-e\-go\-ry,
\index{.infinity1-category@$(\infty,1)$-category}%
it was very convenient to use not merely a \emph{set} of objects with spaces of morphisms between them, but a \emph{space} of objects incorporating all the equivalences and homotopies between them.
This yields a very well-behaved sort of model for $(\infty,1)$-categories as particular simplicial spaces, which Rezk called \emph{complete Segal spaces}.
\index{complete!Segal space}%
\index{Segal!space}%
One especially good aspect of this model is the analogue of \cref{ct:eqv-levelwise}: a map of complete Segal spaces is an equivalence just when it is a levelwise equivalence of simplicial spaces.

When interpreted in Voevodsky's simplicial\index{simplicial!sets} set model of univalent foundations, our precategories are similar to a truncated analogue of Rezk's ``Segal spaces'', while our categories correspond to his ``complete Segal spaces''.
\index{Segal!category}%
Strict categories correspond instead to (a weakened and truncated version of) what are called ``Segal categories''.
It is known that Segal categories and complete Segal spaces are equivalent models for $(\infty,1)$-categories (see e.g.~\cite{bergner:infty-one}), so that in the simplicial set model, categories and strict categories yield ``equivalent'' category theories---although as we have seen, the former still have many advantages.
However, in the more general categorical semantics of a higher topos,
\index{.infinity1-topos@$(\infty,1)$-topos}%
a strict category corresponds to an internal category (in the traditional sense) in the corresponding 1-topos\index{topos} of sheaves, while a category corresponds to a \emph{stack}.
\index{stack}%
The latter are generally a more appropriate sort of ``category'' relative to a topos.

In Rezk's context, what we have called the ``Rezk completion'' corresponds to fibrant replacement
\index{fibrant replacement}
in the model category for complete Segal spaces.
Since this is built using a transfinite induction argument, it most closely matches our second construction as a higher inductive type.
However, in higher topos models of homotopy type theory, the Rezk completion corresponds to \emph{stack completion},\index{completion!stack}\index{stack!completion} which can be constructed either with a transfinite induction~\cite{jt:strong-stacks} or using a Yoneda embedding \cite{bunge:stacks-morita-internal}.


\sectionExercises

\begin{ex}\label{ex:slice-precategory}
  For a precategory $A$ and $a:A$, define the \define{slice precategory} $A/a$.
  \indexsee{precategory!slice}{category, slice}%
  \indexsee{slice (pre)category}{category, slice}%
  Show that if $A$ is a category, so is $A/a$.
  \indexdef{category!slice}%
\end{ex}

\begin{ex}\label{ex:set-slice-over-equiv-functor-category}
  For any set $X$, prove that the slice category $\uset/X$ is equivalent to the functor category $\uset^X$, where in the latter case we regard $X$ as a discrete category.
\end{ex}

\begin{ex}\label{ex:functor-equiv-right-adjoint}
  \index{adjoint!functor}%
  \index{adjoint!equivalence}%
  Prove that a functor is an equivalence of categories if and only if it is a \emph{right} adjoint whose unit and counit are isomorphisms.
\end{ex}

\begin{ex}\label{ct:pre2cat}
  Define the notion of \define{pre-2-category}.
  \indexdef{pre-2-category}%
  Show that precategories, functors, and natural transformations as defined in \cref{sec:transfors} form a pre-2-category.
  Similarly, define a \define{pre-bicategory}
  \indexdef{pre-bicategory}%
  by replacing the equalities (such as those in \cref{ct:functor-assoc,ct:units}) with natural isomorphisms satisfying analogous coherence conditions.
  Define a function from pre-2-categories to pre-bicategories, and show that it becomes an equivalence when restricted and corestricted to those whose hom-pre\-cat\-egories are categories.
\end{ex}

\begin{ex}\label{ct:2cat}
  Define a \define{2-category}
  \indexdef{2-category}%
  to be a pre-2-category satisfying a condition analogous to that of \cref{ct:category}.
  Verify that the pre-2-category of categories \ucat is a 2-category.
  How much of this chapter can be done internally to an arbitrary 2-category?
\end{ex}

\begin{ex}\label{ct:groupoids}
  Define a 2-category whose objects are 1-types, whose morphisms are functions, and whose 2-morphisms are homotopies.
  Prove that it is equivalent, in an appropriate sense, to the full sub-2-category of \ucat spanned by the \emph{groupoids} (categories in which every arrow is an isomorphism).
\end{ex}

\begin{ex}\label{ex:2strict-cat}
  \index{strict!category}%
  Recall that a \emph{strict category} is a precategory whose type of objects is a set.
  Prove that the pre-2-category of strict categories is equivalent to the following pre-2-category.
  \begin{itemize}
  \item Its objects are categories $A$ equipped with a surjection
    % \footnote{Recall that a function $f:X\to Y$ is a \emph{surjection} if for every $y:Y$, there \emph{merely exists} an $x:X$ such that $f(x)=y$.  This is to be distinguished from a \emph{split surjection}, which has the property that for every $y:Y$ there \emph{exists} an $x:X$ such that $f(x)=y$.}
    $p_A:A_0'\to A_0$, where $A_0'$ is a set.
  \item Its morphisms are functors $F:A\to B$ equipped with a function $F_0':A_0' \to B_0'$ such that $p_B \circ F_0' = F_0 \circ p_A$.
  \item Its 2-morphisms are simply natural transformations.
  \end{itemize}
\end{ex}

\begin{ex}\label{ex:pre2dagger-cat}
  Define the pre-2-category of $\dagger$-categories, which has $\dagger$-struc\-tures on its hom-pre\-cat\-egories.
  Show that two $\dagger$-categories are equal precisely when they are ``unitarily equivalent'' in a suitable sense.
\end{ex}

\begin{ex}\label{ct:ex:hocat}
  Prove that a function $X\to Y$ is an equivalence if and only if its image in the homotopy category of \cref{ct:hocat} is an isomorphism.
  Show that the type of objects of this category is $\trunc1\type$.
\end{ex}

\begin{ex}\label{ex:dagger-rezk}
  Construct the $\dagger$-Rezk completion of a $\dagger$-precategory into a $\dagger$-category, and give it an appropriate universal property.
\end{ex}

\begin{ex}\label{ex:rezk-vankampen}
  \index{van Kampen theorem}%
  \index{theorem!van Kampen}%
  \index{fundamental!groupoid}%
  \index{fundamental!pregroupoid}%
  Using fundamental (pre)groupoids from \cref{ct:fundgpd,ct:rezk-fundgpd-trunc1} and the Rezk completion from \cref{sec:rezk}, give a different proof of van Kampen's theorem (\cref{sec:van-kampen}).
\end{ex}

\begin{ex}\label{ex:stack}
  Let $X$ and $Y$ be sets and $p:Y\to X$ a surjection.
  \begin{enumerate}
  \item Define, for any precategory $A$, the category $\mathrm{Desc}(A,p)$ of \define{descent data}
    \indexdef{descent data}%
    in $A$ relative to $p$.
  \item Show that any precategory $A$ is a \define{prestack}
    \indexdef{prestack}%
    for $p$, i.e.\ the canonical functor $A^X \to \mathrm{Desc}(A,p)$ is fully faithful.
  \item Show that if $A$ is a category, then it is a \define{stack}
    \indexdef{stack}%
    for $p$, i.e.\ $A^X \to \mathrm{Desc}(A,p)$ is an equivalence.
  \item Show that the statement ``every strict category is a stack for every surjection of sets'' is equivalent to the axiom of choice.
    \index{axiom!of choice}%
    \index{strict!category}%
  \end{enumerate}
\end{ex}

% Local Variables:
% TeX-master: "hott-online"
% End:
